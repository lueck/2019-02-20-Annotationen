\documentclass[12pt]{scrartcl}
\usepackage[utf8]{inputenc}
\usepackage[T1]{fontenc}
\usepackage{tgtermes}% Times bzw. Nimbus
\usepackage[polutonikogreek,latin,french,english,ngerman]{babel}
\usepackage{setspace}
\usepackage[german=guillemets]{csquotes}
%\MakeAutoQuote{»}{«}% Anführungs-Zeichen
%\MakeAutoQuote*{›}{‹}% einfache Anführungszeichen

\author{Christian Lück}%
\title{Beispiel-Annotationen}%
\subtitle{Zwischen}

\usepackage[%
style=verbose-ibid,%
%citestyle=verbose-ibid,%
idemtracker=constrict,%
pagetracker=true,%
babel=hyphen,%other*,%
sorting=nty,%
sortlocale=de_DE,%
autopunct=true,%
uniquename=init,%
eprint=true,%false,%
hyperref=false,%
backend=biber]{biblatex}
\bibliography{referatory, meine}

\newcommand*{\see}{\autocap{v}gl\adddot}
\newcommand*{\huif}{\autocap{h}ier und im Folgenden}

\renewcommand{\nametitledelim}{\addcolon\addspace}
\renewcommand{\labelnamepunct}{\addcolon\addspace}
\renewcommand{\newunitpunct}{\addcomma\addspace}
\renewcommand{\subtitlepunct}{\addperiod\addspace}

\DeclareFieldFormat[article]{title}{\enquote{#1}}
\DeclareFieldFormat[bookinbook]{title}{\enquote{#1}}
\DeclareFieldFormat[incollection]{title}{\enquote{#1}}
\DeclareFieldFormat[inproceedings]{title}{\enquote{#1}}
\DeclareFieldFormat[inreference]{title}{\enquote{#1}}
\DeclareFieldFormat{volume}{\ifnumerals{#1}{\bibstring{volume}~#1}{#1}}% volume of a book
\DeclareFieldFormat[article,periodical]{volume}{#1}% volume of a journal


%  Was nicht angezeigt werden soll
\newcommand*{\CleanUpReferatory}{%
  \iffieldequalstr{eprinttype}{googlebooks}{%
    \clearfield{eprint}%
    \clearfield{eprinttype}%
  }{}%
  \clearlist{publisher}%
  \clearfield{eprint}%
  \clearfield{eprinttype}%
}
\newcommand*{\CLtoVerf}{%
  \ifkeyword{CL}{%
    % \savename*{author}{\blx@cluecksname}%
    % \restorename{author}{tigger}%\blx@cluecksname}%
    % \clearname{author}%
   }{}%
}
\newcommand*{\RemoveReferatoryLinks}{%
  \clearfield{url}%
  \clearfield{urldate}%
  \clearfield{eprint}%
  \clearfield{eprinttype}%
}
\AtEveryCitekey{\CleanUpReferatory\CLtoVerf}

\DefineBibliographyStrings{ngerman}{%
  number    = {{H\adddot}},
}

% \vno und \vvno (Bd.) nach dem Vorbild von \pno und \ppno 
\DeclarePageCommands{\pno\ppno\vno\vvno}
\makeatletter
\appto\blx@blxinit{%
  \protected\def\vno{\bibstring{volume}}%
  \protected\def\vvno{\bibstring{volumes}}}
\makeatother


\deffootnote{%
1em% Einzug des Textes ab der 2. Zeile
}{0em% Absatzeinzug bei Fn. über mehrere Absätzen
}{\makebox[1em][l]{\textsuperscript{\thefootnotemark}}}


\KOMAoption{headings}{small}
\KOMAoption{footinclude}{false}% Fußzeile im Rand
\KOMAoption{headinclude}{false}% Kopf im Rand


% Schrift der Überschriften
\setkomafont{disposition}{\normalcolor\bfseries}

\RequirePackage[%
hyperfootnotes=false,%
hyperfigures=false,%
linktocpage=false,%
hidelinks=true%
]{hyperref}

\usepackage{soul}% Gesperrter Text: \so{...}
\newcommand{\lit}{\emph}% Literaturtitel
\newcommand{\term}{\emph}% Begriffe

\makeatletter
\renewcommand*{\maketitle}{%
  {\centering
    {\large\@title\par}
    \vskip .5\baselineskip
    {\strut\@author\par}
  }
}
\makeatother

\widowpenalty = 10000
\clubpenalty = 10000

\KOMAoption{draft}{true}% debug overfull boxes
\setlength{\emergencystretch}{3em}

\begin{document}

\onehalfspacing

\maketitle

\noindent%
Die Digitalisierung lässt Forschungsfragen entstehen, die einer
empirischen Revision klassischer Forschungsthemen das Wort reden und
bisweilen sogar neue Perspektiven eröffnen. Technisch getragen werden
solche Vorhaben oft von Annotationen. Berichten möchte ich von der
Annotationspraxis im DFG-Projekt \emph{Das Beispiel im Wissen der
  Ästhetik (1750-1850). Erforschung und Archivierung einer diskursiven
  Praxis}, das von der Hypothese ausgeht, dass die Beispiele einen
wesentlichen Bestandteil des Wissens der philosophischen Ästhetik
ausmachen und dass man eine Wissensgeschichte nicht nur der abstrakten
Begriffe und Ideen (etwa Schönheiten der Natur), sondern auch der
Beispiele (z.B. eine Rose) schreiben muss. Dazu müssen die Beispiele
auch erfasst werden, und zwar vorzüglich in Annotationen.

Das Beispiel ist eine semantische Struktur; zudem gibt es mit »e.g.«
bzw. »z.B.« (in älteren Varianten »z.E.«) eine schon alte
Standardabkürzung, die es auf der Textoberfläche markieren kann, aber
nicht muss; auch syntaktisch Merkmale lassen sich beschreiben, sind
aber so vielfältig wie die Semantik, die sich längst nicht auf das
Verhältnis von Besonderem zu Allgemeinem beschränkt.

Zunächst steht zur Erfassung von Beispielen die ›manuelle‹ Annotation
zur Verfügung. Wünschenswert ist die technische Realisation der
Annotationen als externes Markup zu einem Quell-Dokument, welches in
einem Standard-Format, etwa TEI-XML vorliegt. So lassen sich
Annotationen kollaborativ herstellen und Kontrollannotationen
desselben Texts durch eine weitere Annotatorin werden
realisierbar. Dadurch dass das externe Markup wieder ins TEI-Markup
integriert werden kann, wird der ›Doppelgesichtigkeit‹ digital
edierter Texte Rechnung getragen. Das Auszeichnen im \emph{plain
  text}, welches technisch viel einfacher wäre, kommt hingegen nicht
in Frage.

Das ›manuelle‹ Auszeichnen semantischer Strukturen ist aufwendig (und
die Arbeit der Hand ist dabei das Geringste). Eine automatisiertes
Information Retrieval, ein maschinelles Annotieren von Beispielen wäre
wünschenswert. Allerdings ist diese Aufgabe komplex. Man wird
computerlinguistisch erzeugte Daten (Segmentierungen in Tokens und
Sentences, grammatische Grundformen, Part-Of-Speech-Tags, Tree-Banks),
wie sie vom
\href{https://weblicht.sfs.uni-tuebingen.de/weblichtwiki/index.php/Main_Page}{WebLicht-Service}
(oder von NLP-Bibliotheken) für Eingabe-Dokumente zurückgeliefert
werden, auswerten müssen. Wie jedoch bezieht man die unterschiedlichen
Typen von Markup -- editorisches XML/TEI, das ›manuell‹ erstellte
semantische Markup und dann das WebLicht-Standoff-Markup --
aufeinander? Es stellen sich erst einmal rein technische Probleme, die
noch nicht gelöst sind…% der Bezugsgröße: Das
% WebLicht-XML-Format (TCF) definiert zwar einen Text-Structure-Layer,
% der das im Eingabe-Dokument enthaltene Markup auf die Tokens bezieht,
% allerdings ist dafür bislang kein Parser realisiert worden (was
% m.E. auch an der Verkettungsstruktur der WebLicht-Services liegt).

Die Aussicht, die Auszeichnung von Beispielen zu automatisieren hat
über das Technische hinaus wiederum Rückwirkungen auf die ›manuelle‹
Annotationspraxis. So führen prognostische Überlegungen, welche Art
von Beispielen überhaupt durch einen Algorithmus erfasst werden können
(zunächst nur solche, die auf der Textoberfläche mit »z.B.« markiert
sind) zu einer Auswahl ›manuell‹ annotierter Beispiele; auch versucht
man womöglich, die oft recht komplexen semantischen Verhältnisse in
ihrer Komplexität zu reduzieren. Natürlich ergeben sich bereits ohne
Rücksicht auf eine zukünftige maschinelle Annotation Spannungen
zwischen der vielfältigen Wirklichkeit der Beispiele und dem, was in
einem Annotationsschema (unseres ist in OWL realisiert) bzw. nach den
Annotationsrichtlinien ausdrückbar ist. So ergeben sich Spannungen in
einem Dreieck aus interpretativer Arbeit am philosophischem Text,
Ausdrucksmächtigkeit des Annotationsschemas und künftigem
Algorithmus. Von diesem Spannungen möchte ich im Vortrag berichten und
sie nach Möglichkeit systematisieren.


\newpage

\newpage
Christian Lück ist wissenschaftlicher Mitarbeiter am Lehrgebiet für
Neuere deutsche Literaturwissenschaft und Medienästhetik der
Fernuniversität in Hagen (Prof. Dr. Michael Niehaus) sowie im
DFG-Projekt \emph{Das Beispiel im Wissen der Ästhetik
  (1750-1850). Erforschung und Archivierung einer diskursiven
  Praxis}. Er hat sich an der Ruhr-Universität Bochum mit einer Arbeit
über die doppelte Germanistik sowohl als juristische als auch
philologische Disziplin in der ersten Hälfte des 19. Jahrhunderts
promoviert (\emph{»Ein Nationalunglück, welches der Patriot nur
  beklagen kann«. Der Widerstand gegen Römisches Recht in Deutschland
  1770-1850. München: Fink [im Erscheinen]}). Zum Thema der Tagung:
»Beispiele annotieren: Was und wie viel?« In: \emph{z.B. Zeitschrift
  zum Beispiel} 1 (2018).


\end{document}

%%% Local Variables: 
%%% mode: latex
%%% TeX-master: t
%%% End: 

