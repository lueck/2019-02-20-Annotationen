% Ausgabe als odt mit: mk4ht oolatex Wuppertal_en.tex
\documentclass[12pt]{scrartcl}
\usepackage[utf8]{inputenc}
\usepackage[T1]{fontenc}
\usepackage{tgtermes}% Times bzw. Nimbus
\usepackage[polutonikogreek,latin,french,ngerman,english]{babel}
\usepackage{setspace}
\usepackage{csquotes}
\MakeAutoQuote{»}{«}% Anführungs-Zeichen
\MakeAutoQuote*{›}{‹}% einfache Anführungszeichen

\author{Christian Lück}%
\title{Annotating examples}%
\subtitle{Between hermeneutics, hand-made and machine-made annotations}

\usepackage[%
style=verbose-ibid,%
%citestyle=verbose-ibid,%
idemtracker=constrict,%
pagetracker=true,%
babel=hyphen,%other*,%
sorting=nty,%
sortlocale=en_En,%
autopunct=true,%
uniquename=init,%
eprint=true,%false,%
hyperref=false,%
backend=biber]{biblatex}
\bibliography{referatory, meine}

\newcommand*{\see}{\autocap{v}gl\adddot}
\newcommand*{\huif}{\autocap{h}ier und im Folgenden}

\renewcommand{\nametitledelim}{\addcolon\addspace}
\renewcommand{\labelnamepunct}{\addcolon\addspace}
\renewcommand{\newunitpunct}{\addcomma\addspace}
\renewcommand{\subtitlepunct}{\addperiod\addspace}

\DeclareFieldFormat[article]{title}{\enquote{#1}}
\DeclareFieldFormat[bookinbook]{title}{\enquote{#1}}
\DeclareFieldFormat[incollection]{title}{\enquote{#1}}
\DeclareFieldFormat[inproceedings]{title}{\enquote{#1}}
\DeclareFieldFormat[inreference]{title}{\enquote{#1}}
\DeclareFieldFormat{volume}{\ifnumerals{#1}{\bibstring{volume}~#1}{#1}}% volume of a book
\DeclareFieldFormat[article,periodical]{volume}{#1}% volume of a journal


%  Was nicht angezeigt werden soll
\newcommand*{\CleanUpReferatory}{%
  \iffieldequalstr{eprinttype}{googlebooks}{%
    \clearfield{eprint}%
    \clearfield{eprinttype}%
  }{}%
  \clearlist{publisher}%
  \clearfield{eprint}%
  \clearfield{eprinttype}%
}
\newcommand*{\CLtoVerf}{%
  \ifkeyword{CL}{%
    % \savename*{author}{\blx@cluecksname}%
    % \restorename{author}{tigger}%\blx@cluecksname}%
    % \clearname{author}%
   }{}%
}
\newcommand*{\RemoveReferatoryLinks}{%
  \clearfield{url}%
  \clearfield{urldate}%
  \clearfield{eprint}%
  \clearfield{eprinttype}%
}
\AtEveryCitekey{\CleanUpReferatory\CLtoVerf}

\DefineBibliographyStrings{ngerman}{%
  number    = {{H\adddot}},
}

% \vno und \vvno (Bd.) nach dem Vorbild von \pno und \ppno 
\DeclarePageCommands{\pno\ppno\vno\vvno}
\makeatletter
\appto\blx@blxinit{%
  \protected\def\vno{\bibstring{volume}}%
  \protected\def\vvno{\bibstring{volumes}}}
\makeatother


\deffootnote{%
1em% Einzug des Textes ab der 2. Zeile
}{0em% Absatzeinzug bei Fn. über mehrere Absätzen
}{\makebox[1em][l]{\textsuperscript{\thefootnotemark}}}


\KOMAoption{headings}{small}
\KOMAoption{footinclude}{false}% Fußzeile im Rand
\KOMAoption{headinclude}{false}% Kopf im Rand


% Schrift der Überschriften
\setkomafont{disposition}{\normalcolor\bfseries}

\RequirePackage[%
hyperfootnotes=false,%
hyperfigures=false,%
linktocpage=false,%
hidelinks=true%
]{hyperref}

\usepackage{soul}% Gesperrter Text: \so{...}
\newcommand{\lit}{\emph}% Literaturtitel
\newcommand{\term}{\emph}% Begriffe

\makeatletter
\renewcommand*{\maketitle}{%
  {\centering
    {\large\@title\par}
    %\vskip .5\baselineskip
    {\strut\@subtitle\par}
    \vskip .5\baselineskip
    {\strut\@author\par}
    \vskip .5\baselineskip
  }
}
\makeatother

\widowpenalty = 10000
\clubpenalty = 10000

\KOMAoption{draft}{true}% debug overfull boxes
\setlength{\emergencystretch}{3em}

\begin{document}

\onehalfspacing

\maketitle

\noindent%
In my lecture, I want to give an account of the use and production of
annotations in a research project called
\emph{\foreignlanguage{ngerman}{Das Beispiel im Wissen der Ästhetik
    (1750-1850). Erforschung und Archivierung einer diskursiven
    Praxis}} which is funded by \foreignlanguage{ngerman}{Deutsche
  Forschungsgemeinschaft}.  The project explores the usage and theory
of \emph{examples} in German-language philosophy of aesthetics and
covers the time from about 1750 up to 1850, i.\,e.\ from the founder
of this philosophical discipline, Gottlob Alexander Baumgarten, to
some successors of Hegel, like Karl Rosenkranz.  The project's
hypothesis is that examples (the rose, the tulip, the Alps, etc.)\ are
essential to the foundation of the philosophy of aesthetics and that
one has to care about them--and not only about abstract concepts and
ideas--if one wants to write a history of knowledge.  In fact, our
hypothesis is a broader one and not just constrained to the philosophy
of aesthetics: We do think, that every scientific discipline (every
discourse) somehow depends on the usage of examples, but to a large
extent the foundational role of examples remains shadowed. So the
project's aim is a review of the philosophy of aesthetics by switching
the focus from abstract ideas to the examples and the concrete things
exhibited in them. It wants to be as material-based and empirical as
possible. That requires us to \emph{record} examples.

The example is a semantic structure, and there is an old standardized
abbreviation, »e.\,g.«, which may serve as a \emph{surface marker} but
needn't be present.  Besides the example and the optional surface
marker, there is a more or less abstract concept, principle or maxim
involved, for which the more or less concrete example is given.  The
relation between these two is not constrained to the ratio of the
particular (das Besondere) and the general (das Allgemeine), but
manifold.

For a long period we had a database for recording examples through a
web form.  However, this database only recorded excerpts and resulted
in collectanea.  In contrast, \emph{annotations} are a first class
recorder for examples for further exploration, because the whole
context is present and the examples can be interconnected by simple
technical means.  Our annotation vocabulary is defined in OWL, and we
use a specially written GNU Emacs mode to create external markup.  But
no matter which approaches to follow, database or annotations, you are
facing the same difficulties when it comes to recording or annotating
an example: You have a more or less distinguished and complex form or
vocabulary on the one hand and the example on the other, which often
does not precisely fit the formalized annotation vocabulary or
database fields.  In getting the sense of an example from a
philosophical text there is hermeneutics involved, and this often
clashes with the formalized approach for recording data.  There may be
a clash of spheres, that is crucial to projects in the digital
humanities.  At least we are used to producing text, not structured
data when reading and trying to understand philosophical books and
literature.

Producing annotations manually takes much time.  It forces you to a
kind of close reading.  To accomplish the aims of the project, we
think about machine-made annotations, i.\,e.\ the identification of
examples by some algorithm.  That also affects manual annotation.  The
focus changes when human-made annotations will be used as test cases
(or training data) for algorithms.  There will be no hermeneutics
involved in that algorithm, that's for sure, and it will at best be
able to annotate just a small subset of the phenomena a human
annotator can describe with the formalized vocabulary.  Even the
vocabulary must be adapted to make test cases.

The lecture will explore this triangle between hermeneutics,
human-made, and machine-made annotations. Therefore I will present
some examples, their annotations and an approach to machine-made
annotations of examples.


\newpage
\noindent
Christian Lück is research assistant in the field of modern German
literature and aesthetics of the media at the University of Hagen,
Germany, and also in the research project \emph{Das Beispiel im Wissen
  der Ästhetik (1750-1850). Erforschung und Archivierung einer
  diskursiven Praxis}. In his Ph.\,D. thesis he explored the twofold
Germanistik during the first half of the 19th century, which was a
sub-discipline of law and an emerging philological
discipline. (\emph{»Ein Nationalunglück, welches der Patriot nur
  beklagen kann«. Der Widerstand gegen Römisches Recht in Deutschland
  1770-1850. München: Fink [im Erscheinen]}) Publication on the topic
of the workshop: »Beispiele annotieren: Was und wie viel?« In:
\emph{z.\,B. Zeitschrift zum Beispiel} 1 (2018).

\end{document}

%%% Local Variables: 
%%% mode: latex
%%% TeX-master: t
%%% End: 

