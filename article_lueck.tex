\documentclass{article}

%\usepackage[utf8]{inputenc}%(only for the pdftex engine)
\RequirePackage[no-math]{fontspec}[2017/03/31]%(only for the luatex or the xetex engine)
\usepackage[polutonikogreek,english,ngerman]{babel}
\usepackage[small]{dgruyter}
\usepackage{microtype}

\setmonofont[
  AutoFakeSlant,
  BoldItalicFeatures={FakeSlant},
]{Inconsolatazi4}

\usepackage{xfrac}
%\usepackage{afterpage}
\usepackage{tikz}
\usepackage[export]{adjustbox}

\usepackage[%
style=authoryear,%
autolang=other,%
sorting=nty,%
sortlocale=de_DE,%
autopunct=true,%
uniquename=init,%
eprint=false,%
isbn=false,%
hyperref=false,%
backend=biber]{biblatex}
%% TODO
%\bibliography{Beitrag.bib}
\bibliography{referatory,meine}
%\documentclass{article}

\usepackage[utf8]{inputenc}%(only for the pdftex engine)
%\RequirePackage[no-math]{fontspec}[2017/03/31]%(only for the luatex or the xetex engine)
\usepackage[polutonikogreek,english,ngerman]{babel}
\usepackage[small]{dgruyter}
\usepackage{microtype}

\usepackage{xfrac}
\usepackage{afterpage}

\usepackage[%
style=authoryear,%
autolang=other,%
sorting=nty,%
sortlocale=de_DE,%
autopunct=true,%
uniquename=init,%
eprint=false,%
isbn=false,%
hyperref=false,%
backend=biber]{biblatex}
\bibliography{referatory,meine}

%  Was in bibliographischen Angaben nicht angezeigt werden soll:
\newcommand*{\CleanUpReferatory}{%
  \iffieldequalstr{eprinttype}{googlebooks}{%
    \clearfield{eprint}%
    \clearfield{eprinttype}}{}}%
\newcommand*{\CLclean}{%
  \ifkeyword{CL}{%
    \clearfield{addendum}}{}}%
\AtEveryCitekey{\CleanUpReferatory\CLclean}
\AtEveryBibitem{\CleanUpReferatory\CLclean}%

% Bibliographischer Stil:
\renewcommand*{\labelnamepunct}{\adddot\space}%
\renewbibmacro*{publisher+location+date}{%
  \printlist{publisher}%
  \setunit*{\addcomma\space}%
  \printlist{location}%
  \setunit*{\addspace}%
  \usebibmacro{date}%
  \newunit}%
\DeclareFieldFormat[article]{title}{#1}%
\DeclareFieldFormat[incollection]{title}{#1}%
\DeclareFieldFormat*{pages}{#1}%
\renewcommand*{\bibpagespunct}{\addcolon}%
\renewbibmacro{in:}{}%
\usepackage{xpatch}
\xpatchbibmacro{date+extradate}{%
  \printtext[parens]%
}{%
  \setunit*{\adddot\space}%
  \printtext%
}{}{} 

%% spezielle Macros fuer diesen Text
\newcommand*{\lit}{\textit}%
\newcommand*{\vokabular}{\texttt}%
\newcommand*{\englisch}[1]{\foreignlanguage{english}{\textit{#1}}}%
\newcommand*{\latein}{\textit}%
\newcommand*{\som}{\texttt{standoff-mode}}%
\renewcommand*{\see}{\autocap{v}gl\adddot}%

\begin{document}

  %\articletype{...}% FIXME

\author*[1]{Christian Lück}%
\runningauthor{Christian Lück}%
\affil[1]{FernUniversität in Hagen}%
\title{Beispiele Annotieren}%
\runningtitle{Beispiele Annotieren}%
\subtitle{Zwischen interpretativer Arbeit am philosophischen Text,
  Formalisierung und maschinellen Verfahren}%
\abstract{Dieser Beitrag behandelt Annotationen von Beispielen in
  Texten zur philosophischen Ästhetik. Der Zweck der Annotationen ist
  die Erzeugung eines größeren Datensatzes, mit dem diskursive Trends
  auf einer Zeitleiste darstellbar sind. Es werden zwei verschiedene
  Ideen bezüglich Annotationen vorgestellt. Die eine sieht das
  Annotieren im engen Zusammenhang mit der interpretativen Arbeit am
  philosophischen Text und lässt es zu einer aufwendigen Übung im
  \englisch{close reading} werden. Die andere löst sich davon, die
  Einbettung von Beispielen in den Begriffs- und
  Argumentationszusammenhang eines Textes zu annotieren, und setzt
  stattdessen auf ein linguistisches Modell des Beispiels. Dadurch
  werden Annotationsaufgaben in Algorithmen darstellbar, was im
  komplexeren ersten Modell lediglich ein vages Versprechen gewesen
  ist.}%
\keywords{Annotationen, Literaturwissenschaft, Philosophie, ML}%
\classification[PACS]{...}%
\communicated{...}%
\dedication{...}%
\received{...}%
\accepted{...}%
\journalname{...}%
\journalyear{...}%
\journalvolume{..}%
\journalissue{..}%
\startpage{1}%
\aop%
\DOI{...}%

\maketitle



\section{Beispiele}

Die jüngere Forschung hat wiederholt die Rolle von Beispielen für die
Formierung von Wissen
unterstrichen \parencite{Ruchatz2007a,Schaub2010a,CL2013a,ZB}. Bislang
jedoch ist die Untersuchung von Beispielen stets selbst in einem
exemplarischen Modus geblieben: Einzelne Beispiele, von denen mehr
oder weniger plausibel ist, dass sie zentral für eine diskursive
Formation sind, sind detailliert und mit hermeneutischen Methoden oder
in dekonstruktionistischen Lektüren kommentiert worden. Für eine
Forschung, die auf einer größeren Menge von Beispielen basiert,
existiert schlicht keine strukturierte Datengrundlage.

Das DFG-Projekt \textit{Das Beispiel im Wissen der Ästhetik
  (1750--1850)} (Fern\-Uni\-versi\-tät in Hagen, Leitung Michael Niehaus)
hat sich einerseits das Ziel gesetzt, die Problemgeschichte der
Ästhetik von ihren Beispielen her neu zu beleuchten. Wie das
funktionieren kann, hat Derrida für Immanuel Kants \lit{Kritik der
  Urteilskraft} vorgemacht, der im Zuge einer die Beispiele
würdigenden Analyse von Kants Begriffssystem konstatiert: »Eine
Paradigmatik der Blume lenkt die dritte
\emph{Kritik}.« \Parencite[107]{Derrida1992} Neben ins Detail von
philosophischen Systemen und Argumentationen gehenden Analysen will
das Hagener Projekt jedoch auch die diskursive Praxis des
Beispielgebens archivieren. Ein Ziel ist die Gewinnung eines
Datensatzes aus einem Korpus von Schriften zur philosophischen
Ästhetik. Ein solcher hätte einen erheblichen Mehrwert für die
diskurstheoretische Erforschung des Beispiels im Sinne einer
Archäologie des Wissens nach \textcite{Fouc1997a}: Er würde die
Forschung in die Lage versetzen, i) ein Verzeichnis der Beispiele zu
präsentieren, ii) historische Längsschnitte zu tätigen, die Einblick
in die Häufigkeit einzelner Beispiele im Untersuchungszeitraum
gewähren -- ihre Emergenz, ihren Boom, ihr Verschwinden~-- und so iii)
gegebenenfalls Trends im philosophischen Diskurs mit anderen Diskursen
zu korrelieren, z.\,B. mit der Reiseliteratur des 18., mit dem
Kolonialdiskurs des 19. Jahrhunderts oder mit der Botanik und
Biologie.

% \section{Kollektaneen, Annotationen und die Perspektive maschinellen Lernens}

Für das Erarbeiten eines Beispieldatensatzes gibt es verschiedene
Ansätze. Bereits vor einigen Jahren wurde begonnen, Beispiele durch
ein Webformular in einer Datenbank aufzunehmen.%
\footnote{\url{http://beispiel.germanistik.rub.de}} %
Neben der Textstelle, in der das Beispiel gegeben wird, und Metadaten
über das Schriftstück ist dabei Folgendes erfasst worden: das Beispiel
selbst, also das eher Konkrete, was als Beispiel angeführt wird;
dasjenige, für das das Beispiel angeführt wird, also das eher
allgemeine Konzept, das mit dem Beispiel illustriert, belegt oder
verständlich gemacht wird; und ein optional vorhandener Marker auf der
Textoberfläche wie »z.\,B.«. Allerdings ist klar, dass die Beispiele
in solch einer Datenbank weitgehend ohne Kontext erfasst werden:
weitgehend ohne den begrifflich-argumentativen Kontext und auch ohne
den Kontext anderer Beispiele. Der Vorteil, dass man auf diese Weise
hinsichtlich der Textauswahl vom Urheberrecht weitgehend
uneingeschränkt bleibt, kann den Nachteil nicht aufwiegen, dass man am
Ende mit bloßen Kollektaneen dasteht. Die Exzerpte werden kaum als
Datensatz dienen, auf dessen Grundlage durch maschinelles Lernen
weitere Beispiele in einem Korpus von Schriften gefunden werden
können.

Aus diesen Gründen setzt das DFG-Projekt \textit{Das Beispiel im
  Wissen der Ästhetik} auf die Annotation von Beispielen in
Volltexten. Allerdings gibt es sehr unterschiedliche Arten und Weisen,
Beispiele zu annotieren: Annotationen, die eher die propositionale
Struktur, in der ein Beispiel gegeben wird, erfassen wollen, und
Annotationen, die eher im Zusammenhang mit linguistischen Kategorien
und mit in Algorithmen darstellbaren Verfahren stehen. Im Folgenden
sollen zunächst die Schwierigkeiten beschrieben werden, die bei der
ersten Art auftreten, um dann das Potential von Annotationen der
zweiten Art zu untersuchen.

\section{Beispiele und Argumentation annotieren}
% FIXME: Formulierung Analog zu 3
%  in der propositionalen Struktur annotieren


In einem literaturwissenschaftlichen Projekt mit einem philosophischen
Gegenstand wie der Ästhetik liegt es zunächst näher, die
begrifflich-propositionale Struktur, in der Beispiele angeführt
werden, in den Blick zu nehmen und zu annotieren. Die Schwerkraft,
welche die Idee der Annotationen in diese Richtung zieht, hat ihren
Grund in theoretischen Auffassungen über das Beispiel. Sie gehen
zurück auf einen älteren philosophischen Diskurs, werden teilweise
durch Definitionen im untersuchten Korpus wiederaufgenommen und
bestätigt und werden in der aktuellen Forschung einer Revision
unterzogen. So steht das Beispiel klassischerweise im Zusammenhang mit
der Beziehung zwischen dem Allgemeinen und dem Besonderen; es kann
einen allgemeinen Satz lediglich widerlegen, aber nicht beweisen, hat
jedoch gleichzeitig einen epistemologischen Wert bei der Induktion. %
(Aristoteles, \emph{Anal. pr.} II\,24; \cite[\see][]{Willer2007a}) %
Dieser epistemologischen Dimension des Beispiels muss man sicher noch
andere \emph{Dimensionen} an die Seite stellen \parencite{CL2013b}:
etwa eine rhetorische Dimension, die auf den Beitrag zielt, den
Beispiele im Hinblick darauf leisten, dass ein Text zu seinem Ziel
kommt und seine Leser*innen überzeugt; oder eine konzeptuelle
Dimension, nach der Beispiele verwendete Begriffe klar machen (was
nicht das Gleiche ist wie widerlegen oder belegen); oder eine
normative Dimension, denn Beispiele vermitteln -- gerade in der
Ästhetik -- auch Konzepte, wie etwas \emph{sein sollte} oder was man
angesichts eines Gegenstandes der Natur oder Kunst \emph{empfinden
  sollte}. -- Im Hinblick auf die Annotationen bewirken solche
theoretischen Überlegungen erst einmal, dass die Idee, semantische und
argumentative Strukturen auszuzeichnen, umso notwendiger erscheint,
sich aber gleichzeitig ein sehr komplexes Annotationsunterfangen
ankündigt. Die Annotationen sollen dann Beispiele so erfassen, dass
sie dem mit Theorie armierten Blick möglichst gerecht werden, wozu die
\emph{formalisierte Beschreibung} eines Phänomens~-- die eine
Annotation ihrer eigenen Idee nach schließlich ist~-- ein Maximum an
Expressivität haben muss.

Eines der technischen Mittel mit der höchsten Experssivität für die
Formulierung von Vokabularen ist die \englisch{Web Ontology Language}
(OWL). Das Vokabular zur Annotation von Beispielen ist in der OWL
realisiert \parencite{CL2018a}: Zur Auszeichnung von Textpassagen
dienen OWL-Klassen (owl:Class), von denen es im wesentlichen vier
gibt: \vokabular{Beispiel}, \vokabular{Marker}, \vokabular{Konzept}
und \vokabular{Kontext}. Um Relationen zwischen ausgezeichneten
Textpassagen zu beschreiben, stehen OWL-Objekteigenschaften
(owl:ObjectProperty) zur Verfügung. So kann in den Annotationen über
einen Marker ausgesagt werden, welches Beispiel er markiert, oder über
ein Beispiel, für welches generellere Konzept es als Beispiel
angeführt wird, oder über ein Beispiel, in welchem begrifflichen,
argumentativen oder theoretischen Kontext es angeführt wird. Solche
der Prädikatenlogik ähnliche Aussagen werden immer als Relation
zwischen zwei ausgezeichneten Textpassagen annotiert, wobei die
Objekteigenschaft die Art der Relation, die zwischen den beiden
Passagen besteht, beschreibt, d.\,h. das Prädikat eines RDF-Tripels
darstellt. Die OWL-Objekteigenschaften des Vokabulars sind in einem
Vererbungsbaum so organisiert, dass z.\,B. alle Eigenschaften, mit
denen eine Relation zwischen einem Beispiel und einem Konzept
beschrieben werden kann, von dem Grund-Prädikat
\vokabular{istBeispielFuer} abgeleitet sind. Dieses Prädikat
implizieren per Vererbung (owl:subPropertyOf) insgesamt 24 weitere
Prädikate zur Beschreibung der Relation zwischen Beispiel und Konzept,
welche wiederum in Teilbäumen organisiert sind. Mit diesem
differenzierten Vokabular an Prädikaten lassen sich die oben
skizzierten Dimensionen des Beispiels (epistemologisch, rhetorisch,
konzeptuell, normativ) einerseits auf formalisierte Art und Weise
beschreiben und andererseits bleibt dabei die elementare Relation
zwischen einem Beispiel und einem generellen Konzept, für das es
angeführt wird, aufgrund der Ableitung vom Grund-Prädikat
\vokabular{istBeispielFuer} stets impliziert. Das ist im Hinblick auf
Abfragen auf dem späteren Datensatz von besonderem Interesse. Darüber
hinaus definiert das Vokabular OWL-Dateneigenschaften
(owl:DatatypeProperty), mit welchen zu einer Textpassage freie
Literale angegeben werden können.

Als Annotationswerkzeug kommt ein im Jahr 2015 selbst entwickeltes
Tool zum Einsatz, das mit dem in OWL definierten Vokabular umgehen
kann. Es basiert auf GNU Emacs, speichert die Annotationen als
Stand-Off-Markup und ermöglicht diskontinuierliches Markup.%
\footnote{\url{http://github.com/lueck/standoff-mode}. Zur Zeit der
  Entwicklung von \som\ standen noch keine Werkzeuge zur Verfügung,
  die diskontinuierliches Markup zusammen mit Relationen zwischen
  ausgezeichneten Textpassagen und eine Annotation in XML-
  bzw. TEI-Dokumenten ermöglicht haben. Das \englisch{BRAT Rapid
    Annotation Tool} kam den Anforderungen am nächsten, verarbeitet
  aber nur schlichte Text-Dateien. Das Datenmodell von CATMA hat
  Relationen noch nicht zugelassen und WebAnno war noch in der
  Entwicklungsphase, genau wie Funktionselemente von TextGrid.\par
  Zur Internalisierung des mit \som\ produzierten externen Markup in
  ein TEI-Dokument kommt ein in Haskell geschriebenes Programm zur
  Anwendung: \url{http://github.com/lueck/standoff-tools}.\par
  Das Vokabular zur Beschreibung von Beispielen ist online unter
  \url{http://github.com/lueck/standoff-mode/arb/arb.owl}.} %

Mit den der Prädikatenlogik ähnlichen Aussagen, die mit solch einem
Vokabular darstellbar sind, kann man die semantisch-propositionale
Struktur von Beispielen schon recht gut annotieren. Wie schnell diese
komplex werden kann, sei an einer Stelle aus Kants dritter
\lit{Kritik} \parencite[\pno\,77\psq\ (§\,17)]{KantKdU} demonstriert:

\begin{quote}
  Schönheit ist die Form der Zweckmäßigkeit eines Gegenstandes, sofern
  sie ohne Vorstellung eines Zweckes an ihm wahrgenommen
  wird. [Fußnote:] Man könnte wider diese Erklärung als Instanz
  anführen, daß es Dinge gibt, an denen man eine zweckmäßige Form
  sieht, ohne an ihnen einen Zweck zu erkennen, z.\,B. die öfter aus
  alten Grabhügeln gezogenen, mit einem Loche als zu einem Hefte,
  versehenen steinernen Geräte; die, ob sie zwar in ihrer Gestalt
  deutlich eine Zweckmäßigkeit verraten, für die man den Zweck nicht
  kennt, darum gleichwohl nicht für schön erklärt werden. Allein, daß
  man sie für ein Kunstwerk ansieht, ist schon genug, um gestehen zu
  müssen, daß man ihre Figur auf irgend eine Absicht und einen
  bestimmten Zweck bezieht. Daher auch gar kein unmittelbares
  Wohlgefallen an ihrer Anschauung. Eine Blume hingegen, z.\,B. eine
  Tulpe, wird für schön gehalten, weil eine gewisse Zweckmäßigkeit,
  die so, wie wir sie beurteilen, auf gar keinen Zweck bezogen wird,
  in ihrer Wahrnehmung angetroffen wird.
\end{quote}

Man sieht sogleich, dass das Annotieren von Beispielen keine leichte
Aufgabe ist. Es gibt in der zitierten Passage zwei Beispiel-Marker
»z.\,B.« an der Text-Oberfläche, aber das ist auch schon alles, was
klar ist. Es ist ein erhebliches Maß an hermeneutischem Aufwand
erforderlich, um eine solche Passage zu annotieren; man muss also
schon ziemlich viel von Kants Argumentation und Begriffsystem
verstehen, um überhaupt sinnvoll etwas zu annotieren. So ist es zwar
klar, dass jeweils hinter den Beispielmarkern »z.\,B.« das Beispiel
folgt. Im zweiten Fall ist das Beispiel »ein Tulpe«, im ersten Fall
»die öfter aus alten Grabhügeln gezogenen, mit einem Loche als zu
einem Hefte, versehenen steinernen Geräte; die, ob sie zwar in ihrer
Gestalt deutlich eine Zweckmäßigkeit verraten, für die man den Zweck
nicht kennt, darum gleichwohl nicht für schön erklärt werden«. Hier
bekommt man gleich einen Eindruck davon, dass selbst die Bestimmung
des Umfangs eines Beispiels Schwierigkeiten machen kann. Die lange
Nominalphrase »die öfter aus alten Grabhügeln gezogenen, mit einem
Loche als zu einem Hefte, versehenen steinernen Geräte« gehört ganz
sicher \emph{ganz} zum Beispiel, denn kein Teil dieser Phrase würden
der intensionalen Bestimmung des Beispiels gerecht: Es geht nicht um
›steinerne Geräte‹, sondern um diejenigen ›steinernen Geräte‹, die man
›in alten Grabhügeln‹ findet und die ›mit einem Loche als zu einem
Hefte versehen sind‹. Ohne diese Bestimmungen spricht man über eine
viel größere Klasse von Gegenständen. Aber man sieht auch gleich, dass
man die Nominalkonstruktion in Relativsätze umformen kann, und dann
entsteht die Frage, ob man den mit einem Semikolon abgetrennten
Relativsatz nicht auch zum Beispiel rechnen will. Andererseits wird in
ihm das Besondere des Beispiels mit dem Allgemeinen des Verhältnisses
zweier Begriffe, um die es geht, nämlich Zweckmäßigkeit und Schönheit,
verworben. Das wäre ein Argument, diesen Relativsatz, nicht mehr zum
Beispiel zu zählen. Im Gegensatz zu den Relativsätzen, die man durch
Auflösung der vorhergehenden Nominalkonstruktion erhält, handelt es
sich um einen nicht-restriktiven Relativsatz, und man hätte gute
Gründe für die Regel, dass nicht-restriktive Relativsätze nicht mehr
zum Beispiel gehören sollen. Wenn man sie dennoch als Beispiel
annotiert, dann müsste man eigentlich auch noch den nächsten Satz zum
Beispiel hinzurechnen, da dort auch noch von den Grabbeigaben die Rede
ist.

Im zweiten Fall von »eine Tulpe« ist die Lage nur auf den ersten Blick
klar. Auf den zweiten Blick bemerkt man, dass bereits »eine Blume« ein
Beispiel ist, das durch »z.\,B. eine Tulpe« noch einmal weiter
konkretisiert wird. Hier bilden also zwei Beispiele eine Art Kaskade
der Konkretisierung. Und man bemerkt weiter, dass es in der gesamten
Fußnote (die bis zum Ende des Zitats reicht) um Beispiele geht, denn
im ersten Satz der Fußnote heißt es, dass man gegen den allgemeinen
Obersatz »als Instanz« existierende Dinge mit bestimmten Eigenschaften
anführen könnte. Eine Instanz gegen etwas anzuführen, ist eine
Beispielpraxis, nämlich die des Gegenbeispiels, also des Widerlegens
einer allgemeinen Aussage mittels Beispielen. Ist dann aber »Dinge
[…], an denen man eine zweckmäßige Form sieht, ohne an ihnen einen
Zweck zu erkennen« bereits ein Gegenbeispiel in seiner allgemeinsten,
abstraktesten Form, die dann in den mit »z.\,B.« markierten Beispielen
konkretisiert wird? Oder handelt es sich um einen Begriff, welcher in
den dann folgenden Beispielen konkretisiert wird?

Um die Relationen zwischen den Beispielen und dem allgemeinen Obersatz
annotieren zu können, muss man noch mehr von der Kantischen
Philosophie verstehen. Es geht hier um die Relation zwischen dem
Geschmacksurteil und der Wahrnehmung von Zwecken.%
\footnote{Zentral für das Begriffssystem und auch für die beiden hier
  besprochenen Beispiele ist die Unterscheidung zwischen freier und
  anhängender Schönheit. Vgl. dazu \textcite{Guesken2018a}.} %
Und in der Fußnote wird der Einwand erhoben (und nach Prüfung
verworfen), dass am Schönen Zwecke nicht
nicht % wichtig: doppeltes nicht
wahrgenommen, sondern nicht erkannt würden. -- Das ist ein großer
Unterschied in der Intentionalität, mit der man sich auf einen
Gegenstand bezieht, und somit im Modus der Wahrnehmung. -- Das erste
Beispiel soll als Beleg für diese Gegenthese herhalten und wird dann
im Satz »Allein …« zurückgewiesen. Das Beispiel der Blume Tulpe ist
nicht ebenfalls ein Gegenbeispiel, sondern ein Beispiel, das gegen die
Gegenthese gerichtet ist, einen Unterschied (in der Intentionalität)
zu dem vorherigen Beispiel markiert und den allgemeinen Obersatz
unterstützt. Es soll noch einmal klar machen, wie die Begriffe gemeint
sind. Das Vokabular zur Annotation nennt das die konzeptuelle
Dimension eines Beispiels. \Parencite[\see\ auch][]{CL2013b} Hier geht
es darum, dass die Blume Tulpe klar machen soll, wie es gemeint ist,
dass sich das Urteil, dass etwas schön sei, nicht auf den Zweck dieses
Gegenstandes bezieht. -- Derridas Verdikt, dass eine »Paradigmatik der
Blume« Kants Analytik des Schönen lenke, erweist sich schon als sehr
instruktiv.

Solche Annotationen der propositionalen Struktur, in der Beispiele
gegeben werden, werden ziemlich komplex. Sie zu erstellen gleicht
einem \englisch{close reading} mit einem formalisierten
Beschreibungsvokabular. Mit der Komplexität steigt noch einmal der
zeitliche Aufwand, den manuelle Annotationen ohnehin schon
erfordern. Bei derartigen Investitionen stellt sich die Frage, was man
überhaupt mit solchen Annotationen hinterher anfangen will -- und
kann. Weil man kaum ein ganzes Korpus ästhetischer Schriften wird
annotieren können, ist die Verwendung als Datensatz, an dem
historische Längsschnitte getätigt werden können, nicht oder nur sehr
eingeschränkt gegeben. Die Abfragemöglichkeiten, die aus einer
Annotation entstehen, wären aber sicher ein wichtiger Grund, den
Aufwand zu betreiben. Wenn dies dann jedoch nur in einem ganz
eingeschränkten Umfang realisiert wird, bleiben die Investitionen
zweifelhaft.

Die zweite wichtige Verwendung wäre die als Datensatz für maschinelles
Lernen, aus dem mittels Regression die Parameter eines Algorithmus
gewonnen werden. Der Algorithmus wäre dann anschließend in der Lage,
selbstständig weitere Beispiele zu identifizieren. Das ist natürlich
immer eine Perspektive, die beim Annotieren eine Rolle spielt und
wegen der man sich um mehrere parallele Annotationen eines Werks und
ihre Deckungsgleichheit bemüht.%
\footnote{Vgl. den Beitrag von Evelyn Gius und Nils Reiter im
  vorliegenden Band.} %
Allerdings zeigt die Projekterfahrung, dass bei Annotationen, die so
komplex sind und in die so viel Verständnis des Begriffssystems
einfließen muss, Deckungsgleichheit kaum herzustellen ist. Das
betrifft nicht nur zwei unterschiedliche Annotator*innen, sondern auch
die Annotationen ein und derselben Person mit einer zeitlichen
Differenz von einer Woche, einem Monat oder einem Jahr.%
\footnote{Über ähnliche Erfahrungen mit eigenen Annotationen berichtet
  auch Willard McCarty in seinem Beitrag im vorliegenden Band.} %
Das ist aber nicht der einzige Punkt, der die Tauglichkeit der
Annotationen für \englisch{machine learning} infrage stellt. Es ist
auch kaum zu erwarten, dass in den überaus komplexen Strukturen, die
man auf diese Art und Weise annotiert, Muster stecken, die so an
anderen Stellen wiedergefunden werden können und dann auch noch
tatsächlich Beispiele sind.

Dies und die Erfahrung, wie schwer übereinstimmende Annotationen zu
erreichen sind, gibt zu bedenken, ob Annotationen überhaupt Daten
sind. Reicht ihre formale Homogenität, die sie aufgrund der Verwendung
digitaler Tools haben, schon aus, um ihnen diesen Status zu
verleihen? %
Vielleicht sollte man Annotationen so handhaben, wie
\emph{Experimentalsysteme} in anderen Wissenschaften gehandhabt
werden, und die Frage ›Sind Annotationen Daten?‹ analog zur Frage, ob
ein experimenteller Aufbau, sobald er überhaupt irgendeinen Output
liefert, von Anfang an und jederzeit schon Daten produziert. Es ist
nämlich doch in anderen Wissenschaften nicht so:
\textcite{Rheinberger2001a} hat den Begriff des Experimentalsystems ja
eingeführt, um zu beschreiben, wie lange es dauert und welche in den
Ergebnissen meist ausgeblendeten Abwege länger verfolgt werden
und welche kaum jemals beschriebene Praxis in einem wissenschaftlichen
Labor in den Experimentalaufbau einfließt und welches Begehren das
Laborhandeln trägt, bis ein Experiment endlich Daten produziert, die
etwas zu einer wissenschaftlich relevanten Frage beitragen. Damit ist
natürlich auch der Begriff der Daten als schlicht Gegebenes infrage
gestellt, denn an Stelle des Gegebenseins rückt mit solch einer
Beschreibung eine Labor-Praxis der Zurichtung in den Blick. Deswegen
ist es womöglich verfrüht, bereits bei der übereinstimmenden
Auszeichnung durch zwei oder mehrere Annotator*innen von Daten zu
sprechen. Ob der Experimentalaufbau funktioniert, entscheidet sich ja
erst dann, wenn die Annotationen mit einem Algorithmus
zusammentreffen. Das soll den Wert von Annotationen nicht in Abrede
stellen. Ohne Experimentalsysteme gäbe es ja schließlich gar keine
Daten. Allerdings scheint als Beschreibung des gegenwärtigen Standes
von auf Annotationen bauenden Projekten in den
literaturwissenschaftlichen Digital Humanities der Begriff der Daten
meist nicht passend: Es handelt sich doch eher um Laborarbeit an
Experimentalsystemen in einem frühen Stadium.

\section{Re-Modellierung: Linguistik des Beispiels}

Der Mehrwert eines größeren Datensatzes wäre, wie eingangs skizziert,
z.\,B. die Möglichkeit historischer Längsschnitte. Für solche ist es
gar nicht so interessant, in welcher propositionalen Struktur ein
Beispiel angeführt wird. Wichtiger ist die Verwendung gleicher
Beispiele, also gleicher Wörter. Ein historischer Längsschnitt wird ja
realisiert durch eine Abbildung der Häufigkeit von Beispielen (Wörtern
oder Wortgruppen) auf einer Zeitleiste.

Das eröffnet die Perspektive für eine Re-Modellierung und drastische
Vereinfachung. Wenn einzelne Wörter oder Wortgruppen, die den
zentralen Teil eines Beispiels ausmachen, in den Blick genommen werden
sollen, dann kann das Annotations- bzw. Beschreibungsvokabular
drastisch vereinfacht werden. Allerdings muss auch ein neuer Term in
das Vokabular eingeführt werden, nämlich der \vokabular{Kopf} eines
Beispiels. Während die Nominalphrase »ein Tulpe« das
\vokabular{Beispiel} ist, ist das Nomen »Tulpe« der \vokabular{Kopf}
dieses Beispiels. Im ersten Beispiel des langen Kant-Zitats sieht man
aber gleich die Schwierigkeit, dass es oft nicht ganz leicht ist, den
Kopf zu bestimmen: »Geräte«, »Hefte«, »Loche« oder »Grabhügeln«? Wenn
der Kopf eine ganze Nominalphrase sein darf, dann könnte man die ganze
Nominalkonstruktion »die öfter aus alten Grabhügeln gezogenen, mit
einem Loche als zu einem Hefte, versehenen steinernen Geräte« als Kopf
ansehen. Aber dies zeigt deutlich die sogleich eintretende
Komplexität, wenn man als Kopf mehr als ein Unigram (N-Gram mit N=1)
zulässt: Aufgrund der Produktionsregeln der Chomsky-Grammatik werden
die Köpfe dann potentiell unendlich lang bzw. komplex. Deswegen soll
zunächst versucht werden, bei der Re-Modellierung mit Unigram-Köpfen
auszukommen. Sollte sich herausstellen, dass dies nicht ausreicht,
dann könnte man ausgehend von einem Unigram die größte Nominalphrase,
in der es steht, zu bestimmen versuchen. Dies allerdings setzt gute
Resultate beim Parsen der Phrasenstruktur voraus, wovon bei
historischen Texten des 18. und 19. Jahrhunderts nicht immer
ausgegangen werden kann.

Also sei der \vokabular{Kopf} eines Beispiels definiert als dasjenige
einzelne Token (Unigram), das für das Beispiel signifikant ist. Es
sollte zu den Tokens gehören, die man mit dem oben beschriebenen
Vokabular (das auf die propositionelle Struktur, in der Beispiele
gegeben werden, zielt) als Beispiel annotiert hätte. Diese Definition
mag zu vage und deshalb unbefriedigend erscheinen. Sie ist aber nur so
unbefriedigend, wie es ebenfalls unbefriedigend ist, im ersten
zitierten Beispiel nur eines der Tokens »Geräte«, »Hefte«, »Loche« und
»Grabhügeln« als Kopf zu annotieren.


\section{Manuelle und maschinelle Annotationen}

Durch ein so reformuliertes und vereinfachtes Modell stellt sich auch
gleich die Frage nach Algorithmen neu, die die Annotation übernehmen
können. Verschiedene Ansätze lassen sich denken. So bekommt man beim
Lesen von Schriften der philosophischen Ästhetik schnell den Eindruck,
dass ein ganzes Arsenal von Tieren und Pflanzen in ihnen als Beispiele
vorkommt: Pferde, Affen, Lerchen, Fledermäuse, Nashörner usw. oder
Rosen, Tulpen, Lilien usw. Man kann davon ausgehen, dass Tiere fast
immer als Beispiele angeführt werden. Die Schriften sind schließlich
keine botanischen Abhandlungen, sondern philosophische Schriften zur
Frage der sinnlichen Wahrnehmung, die sich aus der Alltagswelt und den
Schriften ihrer Zeit an Beispielen bedienen. Aber man wäre ein Esel,
wenn man sagen würde, das Tiere immer Beispiele sind, denn manchmal
kommen Tiere auch in Wendungen uneigentlichen Sprechens vor. Man
könnte also eine Tier- oder Pflanzen-Ontologie verwenden, um Tier- und
Pflanzennamen im Korpus zu suchen. Die Vollständigkeit dieser
Ontologien würde über die Genauigkeit (\englisch{precision}), der
Gebrauch metaphorischer Wendungen in den Texten über die Trefferquote
(\englisch{recall}) entscheiden. Bei einem solchen Verfahren würde man
ein äußeres Wissen, die Ontologie, an den Text anlegen.

Lassen sich auch Verfahren konstruieren, die das Beispielwissen der
Texte selber erheben? Verfahren, die nicht mit einem vorgefertigten,
geschlossenen Wissen lediglich eine bestimmte semantische Teilmenge
(alle Tierbeispiele, alle Pflanzenbeispiele) identifizieren? Im
folgenden soll ein zweistufiges Verfahren vorgestellt
werden. \Parencite[\see\ jetzt auch][]{CL2019d} Es tritt nicht an, um
alle Beispiele in einem Korpus zu identifizieren. Aber es soll
möglichst viele mindestens einmal als Beispiel markierte Beispiele
identifizieren.

In der ersten Phase macht sich das Verfahren zu nutze, dass das
Beispiel eine sprachliche Figur ist, die einen Marker an der
Textoberfläche haben kann. Es ist eine der wenigen Figuren mit einer
Standard-Abkürzung als Oberflächenmarker. Das lateinische »e.\,g.«,
\latein{exempli gratia}, wörtlich \emph{um eines Beispiels willen}
oder \emph{aus Gunst des Beispiels}, ist in viele europäische Sprachen
übersetzt worden;%
\footnote{\Textcite[118]{Cappelli1990a} datiert die Entstehung der
  lateinische Abkürzung auf das 17. Jahrhundert und damit deutlich zu
  spät.} %
im Deutschen war im 18. Jahrhundert noch das »z.\,E.«, zum Exempel,
verbreitet, bevor es durch »z.\,B.« verdrängt worden ist. %
% \footnote{ Die Standard-Abkürzung weist nicht nur eine formale
%   Ähnlichkeit mit der Standard-Abkürzung »i.\,e.« auf, sondern auch
%   eine Ähnlichkeit in der Verwendung, denn beide werden dazu
%   verwendet, Begriff klar und deutlich zu
%   machen. \Parencite[\see][78--80]{CL2018a}} %
Die Standard-Abkürzung hat den schönen Effekt, dass man sich sicher
sein kann, dass, wann immer sie auftaucht, ein Beispiel gegeben
wird. -- Das ist zumindest in Schriften zur Ästhetik so, denn in ihnen
werden zwar Reflexionen über Beispiele und auch über das Geben und den
Gebrauch von Beispielen angestellt, aber keine metasprachlichen
Reflexionen über die sprachliche Form von Beispielen, geschweige denn
über die Verwendung dieses Markers. Das Auftauchen des Markers
»z.\,B.« kann als eindeutiger Indikator für das Vorhandensein eines
Beispiels angenommen werden. Auch andere Marker wie »bspw.« können als
eindeutig angesehen werden; andere hingegen nicht, so etwa »wie«, das
nur gelegentlich ein Beispiel markiert, meist aber als
Vergleichspartikel eingesetzt wird; Ähnliches gilt für »etwa«. Das
Wort »Beispiel« kommt auch in Reflexionen über die eigene oder eine
fremde Beispielpraxis vor.

Die eindeutige Markierung ausnutzend soll in der ersten Phase des
Verfahrens der Kopf des Beispiels zu einem Marker identifiziert
werden. Es soll der Einfachheit halber davon ausgegangen werden, dass
der Kopf im selben Satz steht, wie der Marker. Das in dieser ersten
Phase zu lösende Problem ist ein \emph{Auswahlproblem}: Es stehen
mehrere Tokens als Kandidaten für einen Kopf zur Auswahl; die Auswahl
muss akzeptabel sein.

In der zweiten Phase des Verfahrens sollen anhand der in der ersten
Phase gefundenen Beispiel-Köpfe weitere, nicht markierte Beispiele
identifiziert werden. Für jedes weitere Vorkommen derselben Tokens im
Korpus ist also zu entscheiden, ob es sich um ein Beispiel handelt
oder nicht. Das in dieser zweiten Phase zu lösende Problem ist also
ein \emph{Entscheidungsproblem}.

Mit diesem Verfahren wird man natürlich nicht alle Beispiele
erwischen, sondern höchstens diejenigen, die wenigstens einmal mit
einem eindeutigen Marker an der Textoberfläche markiert sind. Es ist
zu erwarten, dass mit steigender Größe des Korpus, mehr Beispiele und
auch mehr voneinander verschiedene Beispiele gefunden werden, die
absolute Anzahl an gefundenen Beispielen also steigt; aber es ist
nicht unbedingt davon auszugehen, dass der Quotient der gefundenen und
der tatsächlich vorhandenen Beispiele sich mit steigender Korpusgröße
verbessert.

Dass man sich von dem Ziel, alle Beispiele zu finden, verabschiedet,
ist zwar ein gravierender Abstrich. Dafür verspricht aber das
Experimentalsystem mit diesem Verfahren viel eher Daten zu liefern als
bei der Annotation der propositionalen Struktur, in der Beispiele
gegeben werden. Denn nun können für jede Phase des Verfahrens zwei
sehr klare und einfache Annotationsaufgaben gestellt werden:
\begin{description}
\item[Phase 1:] Annotiere in jedem Satz, in dem ein eindeutiger
  Beispiel-Marker vorkommt, den Kopf des Beispiels.
\item[Phase 2:] Annotiere zu jedem Vorkommen der Tokens, die zur Menge
  der Beispiel-Köpfe gehören, ob es sich um ein Beispiel handelt oder
  nicht.
\end{description}
Die Chance ist jetzt viel größer, dass die Annotationen der zwei
Phasen zum Trainieren von jeweils einem ML-Algorithmus geeignet
sind. Für die zweite Phase kann das z.\,B. das
Entscheidungsbaum-Lernen sein \parencite[105--120]{Beierle2014a}, für
die erste Phase ein Algorithmus wie der im Folgenden vorgestellte. In
beiden Fällen ist es dann noch von entscheidender Bedeutung, geeignete
Merkmale zu definieren und bereitzustellen, die für die Unterscheidung
in Phase 1 der Kandidaten und in Phase 2 zwischen Ja und Nein relevant
sind.

\subsection{Phase 1: Auswahl des Kopfes}

Wenn man Annotationen, Merkmale und Algorithmen gleich zusammen
konzipiert, lässt sich während des gesamten Prozesses der Modellierung
abschätzen, wie groß die Erfolgsaussichten sind, also die Aussichten
auf ein funktionierendes Experimentalsystem, in dem ein Algorithmus
brauchbare Daten produziert. Wenn man hingegen weder von einem
geeigneten Merkmalsraum noch von einem Algorithmus eine Vorstellung
hat, dann bleibt die Brauchbarkeit der Annotationen ein vages
Versprechen.

\subsubsection{Merkmalsraum}

In die Definition des Merkmalsraums werden \emph{Beobachtungen und
  Hypothesen} einfließen, die man beim Annotieren machen
bzw. aufstellen kann. Unserer Beobachtung zufolge sind Beispiel-Köpfe
meistens Nomen, seltener Vollverben, manchmal Adjektive, aber nie
Hilfverben, Artikel, Präpositionen usw. Die Wortart wird also schon
einmal zum Merkmalsraum gehören.

Relevant ist auch, wie weit ein Token vom Marker entfernt ist: Ein
näher am Marker stehendes Token ist häufiger ein akzeptabler
Beispiel-Kopf als weiter weg stehende. Man kann die Distanz in Tokens,
die zwischen dem Marker und einem Kandidaten stehen, zählen oder auch
in dazwischen liegenden Kommata. Allerdings sieht man an Kants
verschachtelten Sätzen und Nominalkonstruktionen, dass diese einfachen
Distanzmaße auch trügerisch sein können und eine in Satzkonstituenten
gemessene Distanz noch besser wäre: In einer solchen wäre »Geräte«
genauso weit vom Marker entfernt wie »Grabhügeln«. Die Verwendung
eines solchen Maßes würde allerdings erneut gute Resultate beim Parsen
der Phrasenstruktur voraussetzen, wovon bei historischen Texten des
18. und 19. Jahrhunderts nicht immer ausgegangen werden kann.  

Des Weiteren kann man beim Annotieren beobachten, dass der Marker dem
Beispiel meist, aber nicht immer vorangeht. Die Richtung zum Marker
gehört also auch zum Merkmalsraum.

\begin{figure}[t]
  \ttfamily
  \includegraphics[width=.8\textwidth]{datengrundlage}
  \caption{Die digitale Architektur für das zweistufige Verfahren}
  \label{fig:Architektur}
\end{figure}

\afterpage{\clearpage}

Eine weitere Beobachtung ist, dass die Beispiel-Köpfe Wortformen
(Tokens) sind, die eher selten im Text vorkommen. Auch die zugehörigen
Zitierformen kommen eher selten vor. Diese Beobachtung kann man
machen, wenn man nach Häufigkeit geordnete Frequenztabellen der Tokens
eines Textes studiert. Die Beispielköpfe, die Nomen sind, kommen unter
den Nomen mit geringer bis mittlerer Häufigkeit vor, aber nicht unter
den häufigen Nomen. Die häufigsten Nomen in Kants dritter Kritik sind
philosophische Begriffe: Natur (849 Mal), Begriff (746), Prinzip
(485), Zweck (438), Vernunft (404). Anhand dieser häufigen Wörter
werden Dokumente durch automatische Klassifikation vorgegebenen
Kategorien (Themenfeldern) zugeordnet \parencite{Sebastiani2002a} oder
durch Clustering gruppiert \parencite[195--209]{Heyer2006a}. Die Tier-
oder Pflanzen-Beispiele in Texten der philosophischen Ästhetik führen
nicht dazu, dass ein Text z.\,B. der Botanik zugeordnet wird oder mit
Linnés oder Okens Werken in einem Cluster auftaucht. Ein
linguistischer Grund dafür, dass Beispiele unter den Wörtern mit
geringer bis mittlerer Frequenz auftauchen, könnte in der
Thema-Rhema-Gliederung eines Textes gesucht
werden. \Parencite[784-786]{Bussm1990} An dieser Stelle interessiert
allerdings eher, dass die Frequenz ein relevantes Merkmal darstellt.

Die Merkmale werden im \englisch{pre-processing} des Korpus
generiert. Dafür kommt WebLicht%
\footnote{\url{https://weblicht.sfs.uni-tuebingen.de}} %
zum Einsatz. (Vgl. Abb.\,\ref{fig:Architektur}) Dieser Webservice
segmentiert einen Text nach Sätzen und Tokens, bestimmt zu jedem Token
eine Zitierform (Lemma) und die Wortart
(\englisch{Part-of-Speech-tag}, kurz PoS-tag). Die vorbereiteten Daten
werden in einer relationalen Datenbank abgelegt, deren Herzstück eine
Tabelle ist, in der jedes einzelne Token des Korpus in einer Zeile
repräsentiert wird, wobei die Spalten dieser Zeile die durch das
\englisch{pre-processing} gewonnenen Daten darstellen: die ID des
Tokens, die ID des Satzes, das Lemma, das PoS-tag und dann noch die ID
des Dokuments, mittels derer ein Token -- und damit auch die Beispiele
-- auf Metadaten wie die Autor*in oder das Entstehungsjahr beziehbar
wird. Merkmale aus dem Merkmalsraum, die noch nicht durch das
\englisch{pre-processing} generiert worden sind, werden durch
elementare Abfragen dieser Tabelle gewonnen, etwa die Frequenzen oder
die Distanzen zum Marker.


\subsubsection{Eine Funktion zur Bestimmung des Kopfes}

Formal kann die Aufgabe in Phase 1 folgendermaßen beschrieben
werden. Es ist eine Funktion $k$ zu bestimmen, die den Merkmalsvektor
$v_t\in V$ eines jeden Tokens eines Satzes mit einem eindeutigen
Beispielmarker auf das Interval der reellen Zahl zwischen 0 und 1
abbildet: $k:V \mapsto [0,1]$, wobei $V$ der Merkmalsraum
sei. \emph{Dasjenige Token eines Satzes, für das $k(v_t)$ maximal ist,
  soll der Kopf sein.} Das ist die Form, in der das Auswahlproblem,
das sich in Phase 1 stellt, gelöst werden soll.

In der einfachsten Gestalt könnte $k$ die normierte Summe der
gewichteten Merkmale sein bzw. die Summe von gewichteten Funktionen,
die den Merkmalsvektor auswerten:
\begin{equation}
\label{eq:EinfacheGewichteteSumme}
k(v_t) = \sigma \left( \sum\limits_iw_if_i(v_t) \right)
\end{equation}
wobei $w_i$ die Gewichte und $f_i$ die Auswertfunktionen der Merkmale
seien und $\sigma:\mathbb{R}\mapsto [0,1]$ eine Normierungsfunktion sei
z.\,B. eine Schwanenhalsfunktion. Die Auswertfunktionen der Merkmale
müssen normalisierte Werte liefern, also gelte $f_i:V\mapsto [0,1]$.

Will man nicht nur die einzelnen Merkmale auswerten, sondern
Merkmalskombinationen, dann würde man $k$ als Summe solcher Summen
definieren:
\begin{equation}
  k(v_t) = \sigma\left[ \sum\limits_jw_j\sigma \left( \sum\limits_iw_{ji}f_i(v_t) +b_j\right)\right]
\end{equation}
Dann allerdings ist ein Vielfaches an Gewichten zu bestimmen und noch
zu jeder inneren Summe ein Schwellenwert $b_j$. Deshalb soll die
Definition nach Gleichung \ref{eq:EinfacheGewichteteSumme} ausreichen,
bei der man dann freilich die Normierungsfunktion zur Bestimmung des
Maximums weglassen kann.%
\footnote{Es gibt natürlich alternative Formen für $k$. Eine wäre ein
  Inferenznetzwerk mit
  Sicherheitsfaktoren \parencite[\see][90--96]{Beierle2014a}. Allerdings
  ist es dort schwieriger, Gewichte zu bestimmen.}

Die Gewichte können bei vorliegenden manuellen Annotationen durch
Regression gewonnen werden. Alternativ kann man versuchen, sie grob zu
schätzen, indem man überlegt, welches Merkmal wohl am wichtigsten ist,
welches am zweitwichtigsten usw. und dann noch versucht zu bestimmen,
wievielmal wichtiger das wichtigste Merkmal als die anderen Merkmale
ist. Das läuft dann auf ein hybrides Verfahren für die erste Phase
hinaus:
\begin{enumerate}%[a)]
\item Mit den geschätzten Gewichten generiert man eine Liste von
  Beispielköpfen wie in einem regelbasierten Verfahren.
\item Diese unterzieht man einer Revision
  und ersetzt diejenigen Köpfe, die nicht richtig annotiert worden
  sind.
\item Diese manuell korrigierte Liste verwendet man dann als Zielliste
  anstelle der manuellen Annotationen im Volltext und versucht,
  bessere Gewichte durch Regression zu bestimmen.
\end{enumerate}
Die Liste besteht aus Tupeln aus der ID des Satzes und der ID des
Tokens, welches als Kopf bestimmt worden ist. -- Auch dies würde ich
eine Form von Annotationen nennen; sie weist eine Ähnlichkeit mit
Stand-Off-Markup%
\footnote{\Cite[\see\ z.\,B.][575--579]{TEIP5}, und insbesondere das
  Text Corpus Format (TCF) von WebLicht:
  \url{https://weblicht.sfs.uni-tuebingen.de/weblichtwiki/index.php/The_TCF_Format}.} %
auf.


\subsubsection{Wortart}

Nicht nur die Merkmalsgewichte, sondern auch die Auswertfunktionen der
Merkmale enthalten Parameter, die bestimmt werden müssen. So auch die
Auswertfunktion für das PoS-tag. Auch hier werde mit geschätzten
Parametern begonnen.  Sei $p$ die Komponente des Merkmalsvektors, in
der das PoS-tag gegeben ist, wobei das STTS \parencite{Schiller1999a}
verwendet wird, dann sei $f_{PoS}:V\mapsto [0,1]$ definiert durch:
\begin{equation}
  \label{eq:pos}
  f_{PoS}((\dots,p,\dots)) := \left\{
    \begin{array}{ll}
      1 & \mathrm{falls~} p \in \{\mathtt{NE,FM}\} \\
      0,8 & \mathrm{falls~} p \in \{\mathtt{NN}\} \\
      0,5 & \mathrm{falls~} p \in \{\mathtt{VVINF,}\\ ~ & \mathtt{VVIZU,VVPP}\} \\
      0,4 & \mathrm{falls~} p \in \{\mathtt{VVFIN}\} \\
      0,2 & \mathrm{falls~} p \in \{\mathtt{VMINF}\} \\
      0,1 & \mathrm{falls~} p \in \{\mathtt{VAINF}\} \\
      0 & \mathrm{sonst} \\
    \end{array}\right.
\end{equation}


\subsubsection{Distanz zum Marker}

Sei $d_t$ die Distanz zwischen dem untersuchten Token und dem Marker,
gemessen in dazwischenliegenden Tokens, und $l_t$ die Länge des
Satzes, in dem beide vorkommen, wieder gemessen in Tokens. Sowohl
$d_t$ als auch $l_S$ ist eine Komponente des Merkmalsvektors. Dann sei
die normierte Auswertfunktion $f_{dt}:V\mapsto [0,1]$ definiert durch:
\begin{equation}
  \label{eq:tokendistance}
  f_{dt}((\dots,d_t,l_t,\dots)):=1-\frac{d_t}{l_t} 
\end{equation}

Die Distanz in Kommata $f_{dc}$ werde analog aus $d_c$ und $l_c$
errechnet.


\subsubsection{Richtung}

Auch die Richtung enthält einen Schätzwert. Seien $p_m$ und $p_t$ die
Positionsnummer des Markers bzw. des untersuchten Tokens im Satz und
beides im Merkmalsvektor gegeben. Dann sei $f_{dir}:V\mapsto [0,1]$
definiert durch:

\begin{equation}
  \label{eq:direction}
  f_{dir}((\dots,p_m,p_t,\dots)):= \left\{
    \begin{array}{ll}
      \sfrac{1}{4} & \textrm{~falls~} p_t < p_m\\
      \sfrac{3}{4} & \mathrm{~sonst} \\
    \end{array}\right.
\end{equation}

\subsubsection{Frequenz}

Die im Merkmalsvektor als $h_t$ gegebene absolute Häufigkeit des
Tokens im Dokument soll derart auf das Intervall $[0,1]$ abgebildet
werden, dass ein \latein{hapax legomenon} auf $1$ und häufigere Tokens
auf kleinere Werte abgebildet werden. Die Auswertfunktion für die
Token-Frequenz $f_{tf}:V\mapsto [0,1]$ sei definiert durch:
\begin{equation}
  \label{eq:tokenfreq}
  f_{tf}((\dots,p,\dots,h_t,H,\dots)) := \left\{
    \begin{array}{ll}
      1-c\frac{h_t-1}{H_t-1} & \textrm{falls } f_{PoS}((\dots,p,\dots)) > 0\\
      0 & \mathrm{sonst} \\
    \end{array}\right.
\end{equation}
wobei $H_t$ das Maximum der absoluten Frequenzen all der Tokens im
Dokument sei, für die $f_{PoS}>0$ gilt, die also zur offenen
Wortklasse und nicht zu den Stopp-Wörtern gehören. $H_t$ sei ebenfalls
im Merkmalsvektor gegeben. Die Funktion entspricht damit der
\englisch{augmented normalized term frequency} nach
\textcite[518]{Salton1988a}. $c$ ist dabei ein Parameter, der den
Abfall des Funktionswerts bei steigender Frequenz adjustiert.

Die Auswertfunktion für die Lemma-Frequenz $f_{fl}$ sei analog
definiert.

\subsubsection{Ergebnisse aus Phase 1}

Für die Phase 1 des zweistufigen Verfahrens ist ein Prototyp in der
Programmiersprache R implementiert. Als Gewichte werden $w_{PoS}=3$,
$w_{dt}=2$, $w_{dc}=6$, $w_{tf}=0$ und $w_{lf}=4$ verwendet.

In Immanuel Kants dritter \lit{Kritik} gibt es 51 eindeutige Marker
»z.\,B.«. Das Verfahren liefert dazu folgende Beispiel-Köpfe:%
\footnote{Datengrundlage ist hier die von Gutenberg-DE gescrapte
  Fassung.} %
\begin{quote}
  mihi, Substanz, Bergkristall, Körper, Geister, Rose, Rasenplatzes,
  Walde, Schönheit, Grabhügeln, Tulpe, Größe, Tiere, Kunstprodukten,
  Fuß, Affekten, Gebäude, Zorn, Formen, Tulpen, Farben, Lohn,
  Dichtkunst, Kenntnis, Pferdes, Weib, Genius, Tod, Dichter, Haß,
  Leuten, Linie, Bau, tun, Parabel, Garten, Zirkels, Eigenschaft,
  Flüsse, Haus, Körper, Ungeziefer, Winde, Prädikate, Ursache, Made,
  Wassertiere, Erden, Seele, Ewigkeit, Ewigkeit
\end{quote}

Darunter sind einige Wörter, die man nicht erwarten würde, weil es
sich um allgemeine Konzepte handelt: »Substanz«, »Körper«, »Affekten«
oder »Formen« etwa. Jedoch sind »Substanz«, »Körper« und »Affekten«
korrekt. »Formen« ist nicht korrekt und der Fehler kann auch
identifiziert werden: Er besteht darin, in $f_{dc}$ einen Doppelpunkt
nicht als Satzzeichen, nicht als Komma, gezählt zu haben. Obwohl
»Tiere« korrekt erscheint, hätte besser ein Adjektiv im selben Satz
ausgewählt werden sollen. »Schönheit« ist nicht korrekt, vielmehr
handelt es sich um das Thema der Passage -- und der halben dritten
\lit{Kritik}. Dieses Token kommt 112 Mal vor. Auch hier kann aber die
Ursache des Fehlers identifiziert werden: Ein korrekter Kandidat in
diesem Satz wäre »Wohnhause« gewesen, aber die Lemmatisierung durch
WebLicht funktioniert bei Komposita nicht richtig, sondern gibt dann
stets »unkown« aus. Weil es viele Komposita gibt, ist die Häufigkeit
dieses Lemmas entsprechend hoch und das stark gewichtete Merkmal
$f_{fl}$ niedrig. So hat »Schönheit« das Rennen um das Maximum
gewinnen können und ist als Beispiel-Kopf annotiert worden.

Insgesamt handelt es sich um 5 Fehler in 51 Annotationen. Für einen
Anfang mit derart grob geschätzten Gewichten ist diese Genauigkeit
(\englisch{precision}) ganz akzeptabel; bei der Trefferquote
(\englisch{recall}) kann man davon ausgehen, dass alle Bigramme
»z.\,B.« gefunden und zu jedem ein Kopf
bestimmt worden ist. Aber es ist klar, dass eine bessere Bestimmung
der Gewichte durch ein Regressionsverfahren wünschenswert
wäre. Allerdings wird in der Analyse der Fehler auch deutlich, wie
viel vom \englisch{pre-processing} abhängt.


In Karl Rosenkranz' \lit{Ästhetik des
  Häßlichen} \parencite{Rosenkranz1853a} gibt es 127 eindeutige
Beispielmarker. Dies ist die Liste der durch den Algorithmus
ausgewählten Köpfe: (Aus druck-technischen Gründen ist das lange S
(U+017F) durch ein normales s ersetzt worden.)

\begin{quote}
  Holzschnitten, Künstlers, lyrirische, Immermann's, Schaukel, Typus,
  Amphibien, Schaaf, Skropheln, Knochenauftreibungen, Danaë,
  Iphigenia, Danzig, Nehmen, Raums, Landschaftsmaler, Stellen,
  Betrachten, Quixote, Ornamentik, Aristophanes, Pedantismus, Stellen,
  heißt, Bertram, Index, Sonaten, Wandmalerei, Prutz, Strepsiades,
  Landschaftbilde, Platen, Architekturgemälde, Gestalten, Franzosen,
  Schwert, Maria, Cleopatra, Adam, Büffel, Ode, Frazzen, China,
  Wahrheit, Klenze's, Salzburg, Architektur, Physiognomie, Verwesung,
  Unsymmetrische, Epilepsie, Größe, Leben, Buckligter, Leben, Verse,
  Epik, Kirche, Kopperfield, Uebelmacht, Waldbrand, David, Museum,
  Benjamin, Verhältniß, Tabacksasche, Lexikon, Sport, Vaudeville,
  Reim, Romane, Hyäne, Fröschen, Menschen, Lingam, Held, Sue, Juan,
  Halm's, Kindermord, Victor, Göttin, Statue, Marmorstatue,
  Elephanten, Morolf, Bildung, sagen, Hohenbaum, Jerusalem, Koprolith,
  Heine, beweisen, Elephantiasis, Aristophanes,
  \foreignlanguage{polutonikogreek}{<arma}, %ἁϱμα, %
  Tiberius, Scene, Lüge, Kleist, Banquo, Henriade, Lear, Familie,
  Christi, Grabbe, Nase, Gregorius, Hähne, Molières, Töpfers, Sand,
  Attellanen, Hindubettler, Besitznahme, Kotzebue's, Volksbuch,
  Hauptstelle, Erdbeben, Architektur, Schenke, Unzahl, Posse, Cain.
\end{quote}

An beiden Listen wird klar, wie wünschenswert es wäre, nach der
Bestimmung des Kopfes eine Ausweitung auf eine Phrase zu berechnen.

\subsection{Phase 2: Entscheiden über weitere Vorkommen}

Die Ergebnisse aus Phase 1 unterstreichen, dass die Aufgabe in Phase 2
nicht trivial ist. Besonders die Tokens mittlerer Frequenz stellen
eine Herausforderung dar: Der Begriff »Körper«, der einmal korrekt als
Beispielkopf ausgewählt worden ist, kommt 33 Mal in der dritten Kritik
vor. »Größe« sogar 48 Mal. Nur ein Bruchteil der weiteren Vorkommen
stehen im Zusammenhang mit Beispielen.

Manuelle Annotationen lassen sich für die zweite Phase schnell und
ohne viel Aufwand erstellen. Auch ist das \englisch{decision tree
  learning} ein wirklich guter Kandidat für einen ML-Algorithmus, der
zudem vorteilhafterweise nicht so viele Trainingsdaten benötigt wie
ein Perceptron. Die größte Herausforderung liegt in dieser Phase in
der Definition hinreichend diskriminierender Merkmale. Die Frequenz,
die in Phase 1 besonders relevant gewesen ist, reicht hier höchstens
dazu aus, einen Schwellenwert zu definieren, bei dessen Überschreitung
gleich alle Vorkommen nicht annotiert werden, um im Resultat nicht so
viele falschpositive Annotationen zu haben. Hier seien lediglich
einige Ideen zu einem Merkmalsraum skizziert, deren Implementierung
noch auf sich wartet. Es sei das Dokument, in dem ein Token als
Beispielkopf ausgewählt worden ist, das Quelldokument, und der Satz
der Quellsatz; und es sei das Dokument, in dem ein weiteres Vorkommen
dieses Tokens vorliegt, das Zieldokument, und der dieses Token
enthaltende Satz der Zielsatz. Merkmale könnten dann sein:
\begin{itemize}
\item die Ähnlichkeit zwischen Quell- und Zielsatz, wobei als
  Ähnlichkeitsmaß das Vorkommen gleicher Tokens in Frage kommt;
\item die Ähnlichkeit zwischen Umgebungen, die über Quell- und
  Zielsatz hinausgehen, etwa zwei Sätze davor und dahinter, der
  gesamte Absatz oder das Kapitel;
\item die Ähnlichkeit von Einheiten, die kleiner sind als der Quell-
  und Zielsatz, etwa zwischen PoS-Tag-N-Grams, mit denen die
  Phrasenstruktur approximiert werden kann. Dabei wäre besonders auf
  ähnliche Modifizierer wie Adjektive zu achten;
\item eine größere absolute Häufigkeit des Tokens sowohl im Quell- als
  auch im Zieldokument muss ein Faktor sein, der die
  Wahrscheinlichkeit einer positiven Entscheidung reduziert;
\item gegebenenfalls sind Schwellenwerte für die relative Häufigkeit
  festzulegen, bei deren Überschreitung die Entscheidung bei jedem
  Vorkommen des Tokens negativ ausfällt;
\item wenn das Token mehrmals als Kopf ausgewählt worden ist, kann
  dies auch von Relevanz sein.
\end{itemize}

\section*{Fazit}

Auch wenn das reformulierte Modell für Beispiel-Annotationen noch
nicht vollständig implementiert ist, zeichnet sich doch schon jetzt
mit den akzeptablen Ergebnissen der ersten Phase ab, dass die Chance
auf ein funktionierendes Experimentalsystem ungleich höher ist als mit
den komplexen Annotationen der propositionalen Struktur, in der
Beispiele gegeben werden. Der Grund ist, dass die Annotationen
konzeptionell viel näher an einen Algorithmus rücken, während dieser
bei den komplexen Annotationen ein vages Versprechen einer Zukunft
bleibt. In dem reformulierten Modell sind die Spannungen zwischen der
interpretativen Arbeit am philosophischen Text und dem Formalismus des
Vokabulars, die für die komplexen Annotationen kennzeichnend waren,
weitgehend verschwunden und es rücken Maße wie \englisch{precision}
und \englisch{recall} für die Genauigkeit maschineller Annotationen an
ihre Stelle. Entsprechend sind auch die manuellen Annotationen nicht
mehr mit großen hermeneutischen Investitionen verbunden; Annotieren
ist hier kein \englisch{close reading} mehr, sondern tendenziell ein
technischer Vorgang. In dem beschriebenen hybriden Aufbau, in welchem
die Liste der Beispiel-Köpfe, die das regelbasierte Verfahren
geliefert hat, einer Revision unterzogen wird, um dann die Gewichte
mittels Regression besser zu bestimmen, haben die Annotationen
keinerlei Ähnlichkeit mehr mit dem Markup eines TEI-Dokuments. Sie
bleiben aber dennoch deutlich eine Form des Informationsaustauschs in
der Mensch-Maschine-Kommunikation über einen Text.


\printbibliography

% Definition von Baumgarten zeigt die Komplexität

% mehrere Beispiele zu einem Marker, Beispielreihe.

% Rheinberger

\end{document}



%%% Local Variables:
%%% mode: latex
%%% TeX-master: t
%%% TeX-PDF-mode: t
%%% End:

%  LocalWords:  annotieren Humanities Schwellenwerte




%  Was in bibliographischen Angaben nicht angezeigt werden soll:
\newcommand*{\CleanUpReferatory}{%
  \iffieldequalstr{eprinttype}{googlebooks}{%
    \clearfield{eprint}%
    \clearfield{eprinttype}}{}}%
\newcommand*{\CLclean}{%
  \ifkeyword{CL}{%
    \clearfield{addendum}}{}}%
\AtEveryCitekey{\CleanUpReferatory\CLclean}
\AtEveryBibitem{\CleanUpReferatory\CLclean}%

% Bibliographischer Stil:
\renewcommand*{\labelnamepunct}{\adddot\space}%
\renewbibmacro*{publisher+location+date}{%
  \printlist{publisher}%
  \setunit*{\addcomma\space}%
  \printlist{location}%
  \setunit*{\addspace}%
  \usebibmacro{date}%
  \newunit}%
\DeclareFieldFormat[article]{title}{#1}%
\DeclareFieldFormat[incollection]{title}{#1}%
\DeclareFieldFormat*{pages}{#1}%
\renewcommand*{\bibpagespunct}{\addcolon}%
\renewbibmacro{in:}{}%
\usepackage{xpatch}
\xpatchbibmacro{date+extradate}{%
  \printtext[parens]%
}{%
  \setunit*{\adddot\space}%
  \printtext%
}{}{} 

%% spezielle Macros fuer diesen Text
\newcommand*{\lit}{\textit}%
\newcommand*{\vokabular}{\texttt}%
\newcommand*{\englisch}[1]{\foreignlanguage{english}{\textit{#1}}}%
\newcommand*{\latein}{\textit}%
\newcommand*{\som}{\texttt{standoff-mode}}%
\renewcommand*{\see}{\autocap{v}gl\adddot}%

\begin{document}

  %\articletype{...}% FIXME

\author[1]{Christian Lück}%
\runningauthor{Christian Lück}%
\affil[1]{FernUniversität in Hagen}%
\title{Beispiele Annotieren}%
\runningtitle{Beispiele Annotieren}%
\subtitle{Zwischen interpretativer Arbeit am philosophischen Text,
  Formalisierung und maschinellen Verfahren}%
\abstract{Dieser Beitrag behandelt Annotationen von Beispielen in
  Texten zur philosophischen Ästhetik. Der Zweck der Annotationen ist
  die Erzeugung eines größeren Datensatzes, mit dem diskursive Trends
  auf einer Zeitleiste darstellbar sind. Es werden zwei verschiedene
  Ideen bezüglich Annotationen vorgestellt. Die eine sieht das
  Annotieren im engen Zusammenhang mit der interpretativen Arbeit am
  philosophischen Text und lässt es zu einer aufwendigen Übung im
  \englisch{close reading} werden. Die andere löst sich davon, die
  Einbettung von Beispielen in den Begriffs- und
  Argumentationszusammenhang eines Textes zu annotieren, und setzt
  stattdessen auf ein linguistisches Modell des Beispiels. Dadurch
  werden Annotationsaufgaben in Algorithmen darstellbar, was im
  komplexeren ersten Modell lediglich eine Hoffnung gewesen ist.}%
\keywords{Modelling, Form, Segmentation, Literature, Philosophy, Automation}%
%\classification[PACS]{...}%
%\communicated{...}%
%\dedication{...}%
%\received{...}%
%\accepted{...}%
%\journalname{...}%
%\journalyear{...}%
%\journalvolume{..}%
%\journalissue{..}%
%\startpage{1}%
%\aop%
%\DOI{...}%

\maketitle



\section{Beispiele}

Die jüngere Forschung hat wiederholt die Rolle von Beispielen für die
Formierung von Wissen
unterstrichen \parencite{Ruchatz2007a,Schaub2010a,CL2013a,ZB}. Bislang
jedoch ist die Untersuchung von Beispielen stets selbst in einem
exemplarischen Modus geblieben: Einzelne Beispiele, von denen mehr
oder weniger plausibel ist, dass sie zentral für eine diskursive
Formation sind, sind detailliert und mit hermeneutischen Methoden oder
in dekonstruktionistischen Lektüren kommentiert worden. Für eine
Forschung, die auf einer größeren Menge von Beispielen basiert,
existiert schlicht keine strukturierte Datengrundlage.

Das DFG-Projekt \textit{Das Beispiel im Wissen der Ästhetik
  (1750--1850)} (Fern\-Uni\-versi\-tät in Hagen, Leitung Michael Niehaus)
hat sich einerseits das Ziel gesetzt, die Problemgeschichte der
Ästhetik von ihren Beispielen her neu zu beleuchten. Wie das
funktionieren kann, hat Derrida für Immanuel Kants \lit{Kritik der
  Urteilskraft} vorgemacht, der im Zuge einer die Beispiele
würdigenden Analyse von Kants Begriffssystem konstatiert: »Eine
Paradigmatik der Blume lenkt die dritte
\emph{Kritik}« \Parencite[107]{Derrida1992}. Neben ins Detail von
philosophischen Systemen und Argumentationen gehenden Analysen will
das Hagener Projekt jedoch auch die diskursive Praxis des
Beispielgebens archivieren. Ein Ziel ist die Gewinnung eines
Datensatzes aus einem Korpus von Schriften zur philosophischen
Ästhetik. Ein solcher hätte einen erheblichen Mehrwert für die
diskurstheoretische Erforschung des Beispiels im Sinne einer
Archäologie des Wissens nach \textcite{Fouc1997a}: Er würde die
Forschung in die Lage versetzen, i) ein Verzeichnis der Beispiele zu
präsentieren, ii) historische Längsschnitte zu tätigen, die Einblick
in die Häufigkeit einzelner Beispiele im Untersuchungszeitraum
gewähren -- ihre Emergenz, ihren Boom, ihr Verschwinden~-- und so iii)
gegebenenfalls Trends im philosophischen Diskurs mit anderen Diskursen
zu korrelieren, z.\,B. mit der Reiseliteratur des 18., mit dem
Kolonialdiskurs des 19. Jahrhunderts oder mit der Botanik und
Biologie.

% \section{Kollektaneen, Annotationen und die Perspektive maschinellen Lernens}

Für das Erarbeiten eines Beispieldatensatzes gibt es verschiedene
Ansätze. Bereits vor einigen Jahren wurde begonnen, Beispiele durch
ein Webformular in einer Datenbank aufzunehmen.%
\footnote{\url{http://beispiel.germanistik.rub.de}} %
Neben der Textstelle, in der das Beispiel gegeben wird, und Metadaten
über das Schriftstück ist dabei Folgendes erfasst worden: das Beispiel
selbst, also das eher Konkrete, was als Beispiel angeführt wird;
dasjenige, für das das Beispiel angeführt wird, also das eher
allgemeine Konzept, das mit dem Beispiel illustriert, belegt oder
verständlich gemacht wird; und ein optional vorhandener Marker auf der
Textoberfläche wie »z.\,B.«. Allerdings ist klar, dass die Beispiele
in solch einer Datenbank weitgehend ohne Kontext erfasst werden:
weitgehend ohne den begrifflich-argumentativen Kontext und auch ohne
den Kontext anderer Beispiele. Der Vorteil, dass man auf diese Weise
hinsichtlich der Textauswahl vom Urheberrecht weitgehend
uneingeschränkt bleibt, kann den Nachteil nicht aufwiegen, dass man am
Ende mit bloßen Kollektaneen dasteht. Die Exzerpte werden kaum als
Datensatz dienen, auf dessen Grundlage durch maschinelles Lernen
weitere Beispiele in einem Korpus von Schriften gefunden werden
können.

Aus diesen Gründen setzt das DFG-Projekt \textit{Das Beispiel im
  Wissen der Ästhetik} auf die Annotation von Beispielen in
Volltexten. Allerdings gibt es sehr unterschiedliche Arten und Weisen,
Beispiele zu annotieren: Annotationen, die eher die propositionale
Struktur, in der ein Beispiel gegeben wird, erfassen wollen, und
Annotationen, die eher im Zusammenhang mit linguistischen Kategorien
und mit in Algorithmen darstellbaren Verfahren stehen. Im Folgenden
sollen zunächst die Schwierigkeiten beschrieben werden, die bei der
ersten Art auftreten, um dann das Potential von Annotationen der
zweiten Art zu untersuchen.

\section{Beispiele und Argumentation annotieren}
% FIXME: Formulierung Analog zu 3
%  in der propositionalen Struktur annotieren


In einem literaturwissenschaftlichen Projekt mit einem philosophischen
Gegenstand wie der Ästhetik liegt es zunächst näher, die
begrifflich-propositionale Struktur, in der Beispiele angeführt
werden, in den Blick zu nehmen und zu annotieren. Die Schwerkraft,
welche die Idee der Annotationen in diese Richtung zieht, hat ihren
Grund in theoretischen Auffassungen über das Beispiel. Sie gehen
zurück auf einen älteren philosophischen Diskurs, werden teilweise
durch Definitionen im untersuchten Korpus wiederaufgenommen und
bestätigt und werden in der aktuellen Forschung einer Revision
unterzogen. So steht das Beispiel klassischerweise im Zusammenhang mit
der Beziehung zwischen dem Allgemeinen und dem Besonderen; es kann
einen allgemeinen Satz lediglich widerlegen, aber nicht beweisen, hat
jedoch gleichzeitig einen epistemologischen Wert bei der Induktion %
(Aristoteles, \emph{Anal. pr.} II\,24; \cite[\see][]{Willer2007a}). %
Dieser epistemologischen Dimension des Beispiels muss man sicher noch
andere \emph{Dimensionen} an die Seite stellen \parencite{CL2013b}:
etwa eine rhetorische Dimension, die auf den Beitrag zielt, den
Beispiele im Hinblick darauf leisten, dass ein Text zu seinem Ziel
kommt und seine Leser*innen überzeugt; oder eine konzeptuelle
Dimension, nach der Beispiele verwendete Begriffe klar machen (was
nicht das Gleiche ist wie widerlegen oder belegen); oder eine
normative Dimension, denn Beispiele vermitteln -- gerade in der
Ästhetik -- auch Konzepte, wie etwas \emph{sein sollte} oder was man
angesichts eines Gegenstandes der Natur oder Kunst \emph{empfinden
  sollte}. -- Im Hinblick auf die Annotationen bewirken solche
theoretischen Überlegungen erst einmal, dass die Idee, semantische und
argumentative Strukturen auszuzeichnen, umso notwendiger erscheint,
sich aber gleichzeitig ein sehr komplexes Annotationsunterfangen
ankündigt. Die Annotationen sollen dann Beispiele so erfassen, dass
sie dem mit Theorie armierten Blick möglichst gerecht werden, wozu die
\emph{formalisierte Beschreibung} eines Phänomens~-- die eine
Annotation ihrer eigenen Idee nach schließlich ist~-- eine hohe
Expressivität haben muss.%
\footnote{Expressivität ist ein Begriff aus der Theorie logischer
  Systeme und bezeichnet ein Maß dafür, welche Ideen in einem
  logischen System, z.\,B. der Prädikatenlogik erste Stufe,
  dargestellt werden können; vgl. \cite{Beierle2014a}. Der Begriff ist
  auf die Beschreibungen und den Vergleich von Programmiersprachen
  übertragen worden und meint dort eher den praktischen Aspekt der
  Kürze, Klarheit und Lesbarkeit einer Darstellung;
  vgl. z.\,B. \cite[\pno\,308 u.\,ö.]{Sussman2001a} Übertragen auf
  Annotationen meint der Begriff sowohl die Darstellbarkeit von Ideen
  in einem formalisierten System als auch den praktischen Aspekt der
  Klarheit und Lesbarkeit der Darstellung. Die Darstellbarkeit von
  Ideen wird direkt beeinflusst durch (syntaktische) Arten von
  Ausdrucksmitteln: Klassen für Textpassagen, Attribute für sie,
  Relationen zwischen ihnen, deren Stelligkeit, die Möglichkeit,
  Annotationen zum Gegenstand von Annotationen zu machen
  (Reifikation).}

Eines der technischen Mittel mit hoher Expressivität für die
Formulierung von Vokabularen ist die \englisch{Web Ontology Language}
(OWL). Das Vokabular zur Annotation von Beispielen ist in der OWL
realisiert \parencite{CL2018a}: Zur Auszeichnung von Textpassagen
dienen OWL-Klassen (owl:Class), von denen es im wesentlichen vier
gibt: \vokabular{Beispiel}, \vokabular{Marker}, \vokabular{Konzept}
und \vokabular{Kontext}. Um Relationen zwischen ausgezeichneten
Textpassagen zu beschreiben, stehen OWL-Objekteigenschaften
(owl:ObjectProperty) zur Verfügung. So kann in den Annotationen über
einen Marker ausgesagt werden, welches Beispiel er markiert, oder über
ein Beispiel, für welches generellere Konzept es als Beispiel
angeführt wird, oder über ein Beispiel, in welchem begrifflichen,
argumentativen oder theoretischen Kontext es angeführt wird. Solche
der Prädikatenlogik ähnliche Aussagen werden immer als Relation
zwischen zwei ausgezeichneten Textpassagen annotiert, wobei die
Objekteigenschaft die Art der Relation, die zwischen den beiden
Passagen besteht, beschreibt, d.\,h. das Prädikat eines RDF-Tripels
darstellt. Die OWL-Objekteigenschaften des Vokabulars sind in einem
Vererbungsbaum so organisiert, dass z.\,B. alle Eigenschaften, mit
denen eine Relation zwischen einem Beispiel und einem Konzept
beschrieben werden kann, von dem Grund-Prädikat
\vokabular{istBeispielFuer} abgeleitet sind. Dieses Prädikat
implizieren per Vererbung (owl:subPropertyOf) insgesamt 24 weitere
Prädikate zur Beschreibung der Relation zwischen Beispiel und Konzept,
welche wiederum in Teilbäumen organisiert sind. Mit diesem
differenzierten Vokabular an Prädikaten lassen sich die oben
skizzierten Dimensionen des Beispiels (epistemologisch, rhetorisch,
konzeptuell, normativ) einerseits auf formalisierte Art und Weise
beschreiben und andererseits bleibt dabei die elementare Relation
zwischen einem Beispiel und einem generellen Konzept, für das es
angeführt wird, aufgrund der Ableitung vom Grund-Prädikat
\vokabular{istBeispielFuer} stets impliziert. Das ist im Hinblick auf
Abfragen auf dem späteren Datensatz von besonderem Interesse. Darüber
hinaus definiert das Vokabular OWL-Dateneigenschaften
(owl:DatatypeProperty), mit welchen zu einer Textpassage freie
Literale angegeben werden können.

Als Annotationswerkzeug kommt ein im Jahr 2015 selbst entwickeltes
Tool zum Einsatz, das mit dem in OWL definierten Vokabular umgehen
kann. Es basiert auf GNU Emacs, speichert die Annotationen als
Stand-Off-Markup und ermöglicht diskontinuierliches Markup.%
\footnote{\url{http://github.com/lueck/standoff-mode}. Zur Zeit der
  Entwicklung von \som\ standen noch keine Werkzeuge zur Verfügung,
  die diskontinuierliches Markup zusammen mit Relationen zwischen
  ausgezeichneten Textpassagen und eine Annotation in XML-
  bzw. TEI-Dokumenten ermöglicht haben. Das \englisch{BRAT Rapid
    Annotation Tool}, \url{http://brat.nlplab.org}, kam den
  Anforderungen am nächsten, verarbeitet aber nur schlichte
  Text-Dateien. Das Datenmodell von CATMA, \url{http://catma.de}, hat
  Relationen noch nicht zugelassen und WebAnno
  \parencite{EckartDeCast2016a} war noch in der Entwicklungsphase,
  genau wie Funktionselemente von TextGrid;
  \url{http://textgrid.de}.\par
  Zur Internalisierung des mit \som\ produzierten externen Markup in
  ein TEI-Dokument kommt ein in Haskell geschriebenes Programm zur
  Anwendung: \url{http://github.com/lueck/standoff-tools}.\par
  Das Vokabular zur Beschreibung von Beispielen ist online unter
  \url{http://github.com/lueck/standoff-mode/arb/arb.owl}.} %

Mit den der Prädikatenlogik ähnlichen Aussagen, die mit solch einem
Vokabular darstellbar sind, kann man die semantisch-propositionale
Struktur von Beispielen schon recht gut annotieren. Wie schnell diese
komplex werden kann, sei an einer Stelle aus Kants dritter
\lit{Kritik} \parencite[\pno\,77\psq\ (§\,17)]{KantKdU} demonstriert:

\begin{quote}
  Schönheit ist die Form der Zweckmäßigkeit eines Gegenstandes, sofern
  sie ohne Vorstellung eines Zweckes an ihm wahrgenommen
  wird. [Fußnote:] Man könnte wider diese Erklärung als Instanz
  anführen, daß es Dinge gibt, an denen man eine zweckmäßige Form
  sieht, ohne an ihnen einen Zweck zu erkennen, z.\,B. die öfter aus
  alten Grabhügeln gezogenen, mit einem Loche als zu einem Hefte,
  versehenen steinernen Geräte; die, ob sie zwar in ihrer Gestalt
  deutlich eine Zweckmäßigkeit verraten, für die man den Zweck nicht
  kennt, darum gleichwohl nicht für schön erklärt werden. Allein, daß
  man sie für ein Kunstwerk ansieht, ist schon genug, um gestehen zu
  müssen, daß man ihre Figur auf irgend eine Absicht und einen
  bestimmten Zweck bezieht. Daher auch gar kein unmittelbares
  Wohlgefallen an ihrer Anschauung. Eine Blume hingegen, z.\,B. eine
  Tulpe, wird für schön gehalten, weil eine gewisse Zweckmäßigkeit,
  die so, wie wir sie beurteilen, auf gar keinen Zweck bezogen wird,
  in ihrer Wahrnehmung angetroffen wird.
\end{quote}

Man sieht sogleich, dass das Annotieren von Beispielen keine leichte
Aufgabe ist. Es gibt in der zitierten Passage zwei Beispiel-Marker
»z.\,B.« an der Text-Oberfläche, aber das ist auch schon alles, was
klar ist. Es ist ein erhebliches Maß an hermeneutischem Aufwand
erforderlich, um eine solche Passage zu annotieren; man muss also
Kants Argumentation und Begriffsystem %
verstehen, um überhaupt sinnvoll etwas zu annotieren. So ist es zwar
klar, dass jeweils hinter den Beispielmarkern »z.\,B.« das Beispiel
folgt, aber bereits die Bestimmung seines Umfanges bereitet
Schwierigkeiten. Im zweiten Fall ist das Beispiel »ein Tulpe«, im
ersten Fall »die öfter aus alten Grabhügeln gezogenen, mit einem Loche
als zu einem Hefte, versehenen steinernen Geräte; die, ob sie zwar in
ihrer Gestalt deutlich eine Zweckmäßigkeit verraten, für die man den
Zweck nicht kennt, darum gleichwohl nicht für schön erklärt
werden«. Die lange Nominalphrase »die öfter aus alten Grabhügeln
gezogenen, mit einem Loche als zu einem Hefte, versehenen steinernen
Geräte« gehört ohne Zweifel \emph{ganz} zum Beispiel, denn kein Teil
dieser Phrase würden der intensionalen Bestimmung des Beispiels
gerecht: Es geht nicht um ›steinerne Geräte‹, sondern um diejenigen
›steinernen Geräte‹, die man ›in alten Grabhügeln‹ findet und die ›mit
einem Loche als zu einem Hefte versehen sind‹. Ohne diese Bestimmungen
wäre von einer viel größeren Klasse von Gegenständen die
Rede. Allerdings ist es möglich, die Nominalkonstruktion in
Relativsätze umzuformen, und in der Folge entsteht die Frage, ob der
mit einem Semikolon abgetrennte Relativsatz nicht auch zum Beispiel
zu rechnen wäre. Andererseits wird in ihm das Besondere des Beispiels
mit dem Allgemeinen des Verhältnisses zweier Begriffe, um die es in
diesem Abschnitt geht, verwoben: Zweckmäßigkeit und Schönheit. Das
wäre ein Argument, diesen Relativsatz nicht mehr zum Beispiel zu
zählen. Im Gegensatz zu den Relativsätzen, die durch Auflösung der
vorhergehenden Nominalkonstruktion gewonnen werden, handelt es sich um
einen nicht-restriktiven Relativsatz. Es gibt gute Gründe für die
Regel, dass nicht-restriktive Relativsätze nicht mehr zum Beispiel
gehören sollen. Werden sie dennoch als Beispiel annotiert, dann müsste
konsequenterweise der nächsten Satz ebenfalls zum Beispiel
hinzugerechnet werden, da auch dort von den Grabbeigaben die Rede ist.

Im zweiten Fall von »eine Tulpe« ist die Lage nur auf den ersten Blick
einfacher. Auf den zweiten Blick bemerkt man, dass bereits »eine
Blume« ein Beispiel ist, das durch »z.\,B. eine Tulpe« noch einmal
weiter konkretisiert wird. Hier bilden also zwei Beispiele eine Art
Kaskade der Konkretisierung. Und man bemerkt weiter, dass es in der
gesamten Fußnote (die bis zum Ende des Zitats reicht) um Beispiele
geht, denn im ersten Satz der Fußnote heißt es, dass man gegen den
allgemeinen Obersatz »als Instanz« existierende Dinge mit bestimmten
Eigenschaften anführen könnte. Eine Instanz gegen etwas anzuführen,
ist eine Beispielpraxis, nämlich die des Gegenbeispiels, also des
Widerlegens einer allgemeinen Aussage mittels Beispielen. Ist dann
aber »Dinge […], an denen man eine zweckmäßige Form sieht, ohne an
ihnen einen Zweck zu erkennen« bereits ein Gegenbeispiel in seiner
allgemeinsten, abstraktesten Form, die dann in den mit »z.\,B.«
markierten Beispielen konkretisiert wird? Oder handelt es sich um
einen Begriff, welcher in den dann folgenden Beispielen konkretisiert
wird?

Um die Relationen zwischen den Beispielen und dem allgemeinen Obersatz
annotieren zu können, muss man noch mehr von der Kantischen
Philosophie verstehen. Es geht hier um die Relation zwischen dem
Geschmacksurteil und der Wahrnehmung von Zwecken.%
\footnote{Zentral für das Begriffssystem und auch für die beiden hier
  besprochenen Beispiele ist die Unterscheidung zwischen freier und
  anhängender Schönheit. Vgl. dazu \textcite{Guesken2018a}.} %
Und in der Fußnote wird der Einwand erhoben (und nach Prüfung
verworfen), dass am Schönen Zwecke nicht
nicht % wichtig: doppeltes nicht; würden im Plural: Subjekt ist Zwecke!
wahrgenommen, sondern nicht erkannt würden.~-- Das ist ein großer
Unterschied in der Intentionalität, mit der man sich auf einen
Gegenstand bezieht, und somit im Modus der Wahrnehmung. -- Das erste
Beispiel soll als Beleg für diese Gegenthese herhalten und wird dann
im Satz »Allein …« zurückgewiesen. Das Beispiel der Blume Tulpe ist
nicht ebenfalls ein Gegenbeispiel, sondern ein Beispiel, das gegen die
Gegenthese gerichtet ist, einen Unterschied (in der Intentionalität)
zu dem vorherigen Beispiel markiert und den allgemeinen Obersatz
unterstützt. Es soll noch einmal klar machen, wie die Begriffe gemeint
sind. Das Vokabular zur Annotation nennt das die konzeptuelle
Dimension eines Beispiels. \Parencite[\see\ auch][]{CL2013b} Hier geht
es darum, dass die Blume Tulpe klar machen soll, wie es gemeint ist,
dass sich das Urteil, dass etwas schön sei, nicht auf den Zweck dieses
Gegenstandes bezieht. -- Derridas Verdikt, dass eine »Paradigmatik der
Blume« Kants Analytik des Schönen lenke, erweist sich hier als sehr
instruktiv.

Solche Annotationen der propositionalen Struktur, in der Beispiele
gegeben werden, werden komplex. Sie zu erstellen gleicht einem
\englisch{close reading} mit einem formalisierten
Beschreibungsvokabular. Mit der Komplexität steigt noch einmal der
zeitliche Aufwand, den manuelle Annotationen ohnehin schon
erfordern. Bei derartigen Investitionen stellt sich die Frage, was man
überhaupt mit solchen Annotationen hinterher anfangen will -- und
kann. Weil man kaum ein ganzes Korpus ästhetischer Schriften wird
annotieren können, ist die Verwendung als Datensatz, an dem
historische Längsschnitte getätigt werden können, nicht oder nur sehr
eingeschränkt gegeben. Die Abfragemöglichkeiten, die aus einer
Annotation entstehen, sind aber sicher ein wichtiger Grund, den
Aufwand zu betreiben. Wenn dies dann jedoch nur in einem ganz
eingeschränkten Umfang realisiert wird, bleiben die Investitionen
zweifelhaft.

Die zweite wichtige Verwendung wäre die als Datensatz für maschinelles
Lernen, aus dem mittels Regression die Parameter eines Algorithmus
gewonnen werden. Der Algorithmus wäre dann anschließend in der Lage,
selbstständig weitere Beispiele zu identifizieren. Das ist zwar immer
eine Perspektive, die beim Annotieren eine Rolle spielt und wegen der
man sich um mehrere parallele Annotationen eines Werks und ihre
Deckungsgleichheit bemüht.%
\footnote{Vgl. den Beitrag von Evelyn Gius und Nils Reiter im
  vorliegenden Band.} %
Allerdings zeigt die Projekterfahrung, dass bei Annotationen, die so
komplex sind und in die so viel Verständnis des Begriffssystems
einfließen muss, Deckungsgleichheit kaum herzustellen ist. Das
betrifft nicht nur zwei unterschiedliche Annotator*innen, sondern auch
die Annotationen ein und derselben Person mit einer zeitlichen
Differenz von einer Woche, einem Monat oder einem Jahr.%
\footnote{Über ähnliche Erfahrungen mit eigenen Annotationen berichtet
  auch Willard McCarty in seinem Beitrag im vorliegenden Band.} %
Möglicherweise ist diese Erfahrung jedoch nicht allein durch den
komplexen Gegenstand bedingt, sondern hat eine tiefere Ursache in der
hohen Expressivität der formalen Sprache (hier OWL), mit welcher die
Annotationen erstellt werden: Für formale Sprachen konnte gezeigt
werden, dass mit zunehmender Expressivität Entscheidungsprobleme
schwieriger zu beantworten sind.%
\footcite[bes. \pno\,299\psqq]{Okhotin2005a} %
Für menschliche Interpreten sind Aussagen mit expressiven formalen
Sprachen entsprechend schwierig zu lesen.

Die geringe Deckungsgleichheit ist aber nicht der einzige Punkt, der
die Tauglichkeit der Annotationen für \englisch{machine learning} in
Frage stellt. Es ist auch kaum zu erwarten, dass die überaus komplexen
Strukturen, die man auf diese Art und Weise annotiert, Muster
enthalten, die an anderen Stellen wiedergefunden werden können und
tatsächlich Beispiele sind.

Dies und die Erfahrung, wie schwer übereinstimmende Annotationen zu
erreichen sind, gibt zu bedenken, ob Annotationen überhaupt Daten
sind. Reicht ihre formale Homogenität, die sie aufgrund der Verwendung
digitaler Tools haben, schon aus, um ihnen diesen Status zu
verleihen? %
Vielleicht sollte man Annotationen so handhaben, wie
\emph{Experimentalsysteme} in anderen Wissenschaften gehandhabt
werden, und die Frage ›Sind Annotationen Daten?‹ analog zur Frage, ob
ein experimenteller Aufbau, sobald er überhaupt irgendeinen Output
liefert, von Anfang an und jederzeit schon Daten produziert. Es ist
doch in anderen Wissenschaften eben nicht so:
\textcite{Rheinberger2001a} hat den Begriff des Experimentalsystems
eingeführt, um zu beschreiben, wie lange es dauert und welche in den
Ergebnissen meist ausgeblendeten Abwege länger verfolgt werden und
welche kaum jemals beschriebene Praxis in einem wissenschaftlichen
Labor in den Experimentalaufbau einfließt und welches Begehren das
Laborhandeln trägt, bis ein Experiment endlich Daten produziert, die
etwas zu einer wissenschaftlich relevanten Frage beitragen. Damit ist
nicht zuletzt der Begriff der Daten als schlicht Gegebenes infrage
gestellt, denn an Stelle des Gegebenseins rückt mit solch einer
Beschreibung eine Labor-Praxis der Zurichtung in den Blick. Deswegen
ist es womöglich verfrüht, bereits bei der übereinstimmenden
Auszeichnung durch zwei oder mehrere Annotator*innen von Daten zu
sprechen. Ob der Experimentalaufbau funktioniert, entscheidet sich ja
erst dann, wenn die Annotationen mit einem Algorithmus
zusammentreffen. Das soll den Wert von Annotationen nicht in Abrede
stellen. Ohne Experimentalsysteme gäbe es schließlich keine
Daten. Allerdings scheint als Beschreibung des gegenwärtigen Standes
von auf Annotationen bauenden Projekten in den
literaturwissenschaftlichen Digital Humanities der Begriff der Daten
oft nicht passend: Es handelt sich doch eher um Laborarbeit an
Experimentalsystemen in einem frühen Stadium.

\section{Re-Modellierung: Linguistik des Beispiels}

Der Mehrwert eines größeren Datensatzes ist, wie eingangs skizziert,
z.\,B. die Möglichkeit historischer Längsschnitte. Für solche ist es
gar nicht so interessant, in welcher propositionalen Struktur ein
Beispiel angeführt wird. Wichtiger ist die Verwendung gleicher
Beispiele, also gleicher Wörter. Ein historischer Längsschnitt wird ja
realisiert durch eine Abbildung der Häufigkeit von Beispielen (Wörtern
oder Wortgruppen) auf einer Zeitleiste.

Das eröffnet die Perspektive für eine Re-Modellierung und drastische
Vereinfachung. Wenn einzelne Wörter oder Wortgruppen, die den
zentralen Teil eines Beispiels ausmachen, in den Blick genommen werden
sollen, dann kann das Annotations- bzw. Beschreibungsvokabular stark
vereinfacht werden. Allerdings muss auch ein neuer Term in das
Vokabular eingeführt werden, nämlich der \vokabular{Kopf} eines
Beispiels. Während die Nominalphrase »ein Tulpe« das
\vokabular{Beispiel} ist, ist das Nomen »Tulpe« der \vokabular{Kopf}
dieses Beispiels. Im ersten Beispiel des langen Kant-Zitats wird
jedoch die Schwierigkeit deutlich, dass es oft nicht ganz leicht ist,
den Kopf zu bestimmen: »Geräte«, »Hefte«, »Loche« oder »Grabhügeln«?
Wenn der Kopf eine ganze Nominalphrase sein darf, dann könnte man die
ganze Nominalkonstruktion »die öfter aus alten Grabhügeln gezogenen,
mit einem Loche als zu einem Hefte, versehenen steinernen Geräte« als
Kopf ansehen. Aber dies zeigt deutlich die sogleich eintretende
Komplexität, wenn als Kopf mehr als ein Unigram (N-Gram mit N=1)
zugelassen ist: Aufgrund der Produktionsregeln der Chomsky-Grammatik werden
die Köpfe dann potentiell unendlich lang bzw. komplex. Deswegen soll
zunächst versucht werden, bei der Re-Modellierung mit Unigram-Köpfen
auszukommen. Sollte sich herausstellen, dass dies nicht ausreicht,
dann wäre ausgehend von einem Unigram die größte Nominalphrase, in der
es steht, zu bestimmen. Dies allerdings setzt gute Resultate beim
Parsen der Phrasenstruktur voraus, wovon bei historischen Texten des
18. und 19. Jahrhunderts nicht immer ausgegangen werden kann.

Also sei der \vokabular{Kopf} eines Beispiels definiert als dasjenige
einzelne Token (Unigram), das für das Beispiel signifikant ist. Es
sollte zu den Tokens gehören, die mit dem oben beschriebenen Vokabular
(das auf die propositionelle Struktur, in der Beispiele gegeben
werden, zielt) als Beispiel annotiert worden wäre. Diese Definition
mag zu vage und deshalb unbefriedigend erscheinen. Sie ist aber nur so
unbefriedigend, wie es ebenfalls unbefriedigend ist, im ersten
zitierten Beispiel nur eines der Tokens »Geräte«, »Hefte«, »Loche« und
»Grabhügeln« als Kopf zu annotieren.


\section{Manuelle und maschinelle Annotationen}

Durch ein so reformuliertes und vereinfachtes Modell stellt sich auch
die Frage nach Algorithmen neu, die die Annotation übernehmen
können. Verschiedene Ansätze lassen sich denken. So bekommt man beim
Lesen von Schriften der philosophischen Ästhetik schnell den Eindruck,
dass ein ganzes Arsenal von Tieren und Pflanzen in ihnen als Beispiele
vorkommt: Pferde, Affen, Lerchen, Fledermäuse, Nashörner usw. oder
Rosen, Tulpen, Lilien usw. Es ist davon auszugehen, dass Tiere fast
immer als Beispiele angeführt werden. Die Schriften sind schließlich
keine botanischen Abhandlungen, sondern philosophische Schriften zur
Frage der sinnlichen Wahrnehmung, die sich aus der Alltagswelt und den
Schriften ihrer Zeit an Beispielen bedienen. Aber man wäre ein Esel,
wenn man sagen würde, das Tiere immer Beispiele sind, denn manchmal
kommen Tiere auch in Wendungen uneigentlichen Sprechens vor. Man
könnte also eine Tier- oder Pflanzen-Ontologie verwenden, um Tier- und
Pflanzennamen im Korpus zu suchen. Die Vollständigkeit dieser
Ontologien würde über die Genauigkeit (\englisch{precision}), der
Gebrauch metaphorischer Wendungen in den Texten über die Trefferquote
(\englisch{recall}) entscheiden. Bei einem solchen Verfahren würde
ein äußeres Wissen, die Ontologie, an den Text angelegt.

Lassen sich auch Verfahren konstruieren, die das Beispielwissen der
Texte selber erheben? Verfahren, die nicht mit einem vorgefertigten,
geschlossenen Wissen lediglich eine bestimmte semantische Teilmenge
(alle Tierbeispiele, alle Pflanzenbeispiele) identifizieren? Im
folgenden soll ein zweistufiges Verfahren vorgestellt
werden. \Parencite[\see\ jetzt auch][]{CL2019d} Es tritt nicht an, um
alle Beispiele in einem Korpus zu identifizieren. Aber es soll
möglichst viele mindestens einmal als Beispiel markierte Beispiele
identifizieren.

In der ersten Phase macht sich das Verfahren zu nutze, dass das
Beispiel eine sprachliche Figur ist, die einen Marker an der
Textoberfläche haben kann. Es ist eine der wenigen Figuren mit einer
Standard-Abkürzung als Oberflächenmarker. Das lateinische »e.\,g.«,
\latein{exempli gratia}, wörtlich \emph{um eines Beispiels willen}
oder \emph{aus Gunst des Beispiels}, ist in viele europäische Sprachen
übersetzt worden;%
\footnote{\Textcite[118]{Cappelli1990a} datiert die Entstehung der
  lateinische Abkürzung auf das 17. Jahrhundert und damit deutlich zu
  spät.} %
im Deutschen war im 18. Jahrhundert noch das »z.\,E.«, zum Exempel,
verbreitet, bevor es durch »z.\,B.« verdrängt worden ist. %
% \footnote{ Die Standard-Abkürzung weist nicht nur eine formale
%   Ähnlichkeit mit der Standard-Abkürzung »i.\,e.« auf, sondern auch
%   eine Ähnlichkeit in der Verwendung, denn beide werden dazu
%   verwendet, Begriff klar und deutlich zu
%   machen. \Parencite[\see][78--80]{CL2018a}} %
Die Standard-Abkürzung hat den schönen Effekt, dass es sicher ist,
dass bei ihrem Auftauchen ein Beispiel gegeben wird. -- Das ist
zumindest in Schriften zur Ästhetik so, denn in ihnen werden zwar
Reflexionen über Beispiele und auch über das Geben und den Gebrauch
von Beispielen angestellt, aber keine metasprachlichen Reflexionen
über die sprachliche Form von Beispielen, geschweige denn über die
Verwendung dieses Markers. Das Auftauchen des Markers »z.\,B.« kann
als eindeutiger Indikator für das Vorhandensein eines Beispiels
angenommen werden. Auch andere Marker wie »bspw.« können als eindeutig
angesehen werden; andere hingegen nicht, so etwa »wie«, das nur
gelegentlich ein Beispiel markiert, meist aber als Vergleichspartikel
eingesetzt wird; Ähnliches gilt für »etwa«. Das Wort »Beispiel« kommt
auch in Reflexionen über die eigene oder eine fremde Beispielpraxis
vor.

Die eindeutige Markierung ausnutzend soll in der ersten Phase des
Verfahrens der Kopf des Beispiels zu einem Marker identifiziert
werden. Es soll der Einfachheit halber davon ausgegangen werden, dass
der Kopf im selben Satz steht, wie der Marker. Das in dieser ersten
Phase zu lösende Problem ist ein \emph{Auswahlproblem}: Es stehen
mehrere Tokens als Kandidaten für einen Kopf zur Auswahl; die Auswahl
muss akzeptabel sein.

In der zweiten Phase des Verfahrens sollen anhand der in der ersten
Phase gefundenen Beispiel-Köpfe weitere, nicht markierte Beispiele
identifiziert werden. Für jedes weitere Vorkommen derselben Tokens im
Korpus ist also zu entscheiden, ob es sich um ein Beispiel handelt
oder nicht. Das in dieser zweiten Phase zu lösende Problem ist also
ein \emph{Entscheidungsproblem}.

Mit diesem Verfahren werden nicht alle Beispiele erkannt, sondern
höchstens diejenigen, die wenigstens einmal mit einem eindeutigen
Marker an der Textoberfläche markiert sind. Es ist zu erwarten, dass
mit steigender Größe des Korpus, mehr Beispiele und auch mehr
voneinander verschiedene Beispiele gefunden werden, die absolute
Anzahl an gefundenen Beispielen also steigt; aber es ist nicht davon
auszugehen, dass der Quotient der gefundenen und der tatsächlich
vorhandenen Beispiele sich mit steigender Korpusgröße verbessert.

Dass man sich von dem Ziel, alle Beispiele zu finden, verabschiedet,
ist zwar ein gravierender Abstrich. Dafür verspricht aber das
Experimentalsystem mit diesem Verfahren eher Daten zu liefern als bei
der Annotation der propositionalen Struktur, in der Beispiele gegeben
werden. Denn nun können für jede Phase des Verfahrens zwei sehr klare
und einfache Annotationsaufgaben gestellt werden:
\begin{description}
\item[Phase 1:] Annotiere in jedem Satz, in dem ein eindeutiger
  Beispiel-Marker vorkommt, den Kopf des Beispiels.
\item[Phase 2:] Annotiere zu jedem Vorkommen der Tokens, die zur Menge
  der Beispiel-Köpfe gehören, ob es sich um ein Beispiel handelt oder
  nicht.
\end{description}
Die Chance ist jetzt viel größer, dass die Annotationen der zwei
Phasen zum Trainieren von jeweils einem Machine-Learning-Algorithmus
geeignet sind. Für die zweite Phase kann das z.\,B. das
Entscheidungsbaum-Lernen sein \parencite[105--120]{Beierle2014a}, für
die erste Phase ein Algorithmus wie der im Folgenden vorgestellte. In
beiden Fällen ist es dann noch von entscheidender Bedeutung, geeignete
Merkmale zu definieren und bereitzustellen, die für die Unterscheidung
in Phase 1 der Kandidaten und in Phase 2 zwischen Ja und Nein relevant
sind.

\subsection{Phase 1: Auswahl des Kopfes}

Wenn man Annotationen, Merkmale und Algorithmen gleich zusammen
konzipiert, lässt sich während des gesamten Prozesses der Modellierung
abschätzen, wie groß die Erfolgsaussichten sind, also die Aussichten
auf ein funktionierendes Experimentalsystem, in dem ein Algorithmus
brauchbare Daten produziert. Wenn man hingegen weder von einem
geeigneten Merkmalsraum noch von einem Algorithmus eine Vorstellung
hat, dann bleibt die Brauchbarkeit % Redundanz gewollt
der Annotationen ein Versprechen.

\subsubsection{Merkmalsraum}

In die Definition des Merkmalsraums werden \emph{Beobachtungen und
  Hypothesen} einfließen, die man beim Annotieren machen
bzw. aufstellen kann. Unserer Beobachtung zufolge sind Beispiel-Köpfe
meistens Nomen, seltener Vollverben, manchmal Adjektive, aber nie
Hilfverben, Artikel, Präpositionen usw. Die Wortart gehört also zum
Merkmalsraum.

Relevant ist auch, wie weit ein Token vom Marker entfernt ist: Ein
näher am Marker stehendes Token ist häufiger ein akzeptabler
Beispiel-Kopf als weiter weg stehende. Man kann die Distanz in Tokens,
die zwischen dem Marker und einem Kandidaten stehen, messen
oder auch in dazwischen liegenden Kommata. Allerdings sieht man an
Kants verschachtelten Sätzen und Nominalkonstruktionen, dass diese
einfachen Distanzmaße auch trügerisch sein können und eine in
Satzkonstituenten gemessene Distanz noch besser wäre: In einer solchen
wäre »Geräte« genauso weit vom Marker entfernt wie »Grabhügeln«. Die
Verwendung eines solchen Maßes würde allerdings erneut gute Resultate
beim Parsen der Phrasenstruktur voraussetzen, wovon bei historischen
Texten des 18. und 19. Jahrhunderts nicht immer ausgegangen werden
kann.

Des Weiteren kann man beim Annotieren beobachten, dass der Marker dem
Beispiel meist, aber nicht immer vorangeht. Die Richtung zum Marker
gehört also auch zum Merkmalsraum.

Eine weitere Beobachtung ist, dass die Beispiel-Köpfe Wortformen
(Tokens) sind, die eher selten im Text vorkommen. Auch die zugehörigen
Zitierformen kommen eher selten vor. Diese Beobachtung kann man
machen, wenn man nach Häufigkeit geordnete Frequenztabellen der Tokens
eines Textes studiert. Die Beispielköpfe, die Nomen sind, kommen unter
den Nomen mit geringer bis mittlerer Häufigkeit vor, aber nicht unter
den häufigen Nomen. Die häufigsten Nomen in Kants dritter Kritik sind
philosophische Begriffe: Natur (849 Mal), Begriff (746), Prinzip
(485), Zweck (438), Vernunft (404). Anhand dieser häufigen Wörter
werden Dokumente durch automatische Klassifikation vorgegebenen
Kategorien (Themenfeldern) zugeordnet \parencite{Sebastiani2002a} oder
durch Clustering gruppiert \parencite[195--209]{Heyer2006a}. Die Tier-
oder Pflanzen-Beispiele in Texten der philosophischen Ästhetik führen
nicht dazu, dass ein Text z.\,B. der Botanik zugeordnet wird oder mit
Linnés oder Okens Werken in einem Cluster auftaucht. Ein
linguistischer Grund dafür, dass Beispiele unter den Wörtern mit
geringer bis mittlerer Frequenz auftauchen, könnte in der
Thema-Rhema-Gliederung eines Textes gesucht
werden. \Parencite[784-786]{Bussm1990} An dieser Stelle interessiert
allerdings eher, dass die Frequenz ein relevantes Merkmal darstellt.

\begin{figure}
  \centering
  %\ttfamily
  \includegraphics[max width=.8\textwidth]{abb/datengrundlage.pdf}
  \caption{Die digitale Architektur für das zweistufige Verfahren}
  \label{fig:Architektur}
\end{figure}

Die Merkmale werden im \englisch{pre-processing} des Korpus
generiert. Dafür kommt WebLicht%
\footnote{\url{https://weblicht.sfs.uni-tuebingen.de}} %
zum Einsatz (vgl. Abb.\,\ref{fig:Architektur}). Dieser Webservice
segmentiert einen Text nach Sätzen und Tokens, bestimmt zu jedem Token
eine Zitierform (Lemma) und die Wortart
(\englisch{Part-of-Speech-tag}, kurz PoS-tag). Die vorbereiteten Daten
werden in einer relationalen Datenbank abgelegt, deren Herzstück eine
Tabelle ist, in der jedes einzelne Token des Korpus in einer Zeile
repräsentiert wird, wobei die Spalten dieser Zeile die durch das
\englisch{pre-processing} gewonnenen Daten darstellen: die ID des
Tokens, die ID des Satzes, das Lemma, das PoS-tag und dann noch die ID
des Dokuments, mittels derer ein Token -- und damit auch die Beispiele
-- auf Metadaten wie die Autor*in oder das Entstehungsjahr beziehbar
wird. Merkmale aus dem Merkmalsraum, die noch nicht durch das
\englisch{pre-processing} generiert worden sind, werden durch
elementare Abfragen dieser Tabelle gewonnen, etwa die Frequenzen oder
die Distanzen zum Marker.


\subsubsection{Eine Funktion zur Bestimmung des Kopfes}

Formal kann die Aufgabe in Phase 1 folgendermaßen beschrieben
werden. Es ist eine Funktion $k$ zu bestimmen, die den Merkmalsvektor
$v_t\in V$ eines jeden Tokens eines Satzes mit einem eindeutigen
Beispielmarker auf das Interval der reellen Zahl zwischen 0 und 1
abbildet: $k:V \mapsto [0,1]$, wobei $V$ der Merkmalsraum
sei. \emph{Dasjenige Token eines Satzes, für das $k(v_t)$ maximal ist,
  soll der Kopf sein.} Das ist die Form, in der das Auswahlproblem,
das sich in Phase 1 stellt, gelöst werden soll.

In der einfachsten Gestalt könnte $k$ die normierte Summe der
gewichteten Merkmale sein bzw. die Summe von gewichteten Funktionen,
die den Merkmalsvektor auswerten:
\begin{equation}
\label{eq:EinfacheGewichteteSumme}
k(v_t) = \sigma \left( \sum\limits_iw_if_i(v_t) \right)
\end{equation}
wobei $w_i$ die Gewichte und $f_i$ die Auswertfunktionen der Merkmale
seien und $\sigma:\mathbb{R}\mapsto [0,1]$ eine Normierungsfunktion sei
z.\,B. eine Schwanenhalsfunktion. Die Auswertfunktionen der Merkmale
müssen normalisierte Werte liefern, also gelte $f_i:V\mapsto [0,1]$.

Will man nicht nur die einzelnen Merkmale auswerten, sondern
Merkmalskombinationen, dann würde man $k$ als Summe solcher Summen
definieren:
\begin{equation}
  k(v_t) = \sigma\left[ \sum\limits_jw_j\sigma \left( \sum\limits_iw_{ji}f_i(v_t) +b_j\right)\right]
\end{equation}
Dann allerdings ist ein Vielfaches an Gewichten zu bestimmen und noch
zu jeder inneren Summe ein Schwellenwert $b_j$. Deshalb soll die
Definition nach Gleichung \ref{eq:EinfacheGewichteteSumme} ausreichen,
bei der man dann freilich die Normierungsfunktion zur Bestimmung des
Maximums weglassen kann.%
\footnote{Es gibt natürlich alternative Formen für $k$. Eine wäre ein
  Inferenznetzwerk mit
  Sicherheitsfaktoren \parencite[\see][90--96]{Beierle2014a}. Allerdings
  ist es dort schwieriger, Gewichte zu bestimmen.}

Die Gewichte können bei vorliegenden manuellen Annotationen durch
Regression gewonnen werden. Alternativ kann man versuchen, sie grob zu
schätzen, indem man überlegt, welches Merkmal wohl am wichtigsten ist,
welches am zweitwichtigsten usw. und versucht zu bestimmen,
wievielmal wichtiger das wichtigste Merkmal als die anderen Merkmale
ist. Das läuft dann auf ein hybrides Verfahren für die erste Phase
hinaus:
\begin{enumerate}%[a)]
\item Mit den geschätzten Gewichten generiert man eine Liste von
  Beispielköpfen wie in einem regelbasierten Verfahren.
\item Diese unterzieht man einer Revision
  und ersetzt diejenigen Köpfe, die nicht richtig annotiert worden
  sind.
\item Diese manuell korrigierte Liste verwendet man dann als Zielliste
  anstelle der manuellen Annotationen im Volltext und versucht, die
  Gewichte durch Regression akkurater zu bestimmen.
\end{enumerate}
Die Liste besteht aus Tupeln aus der ID des Satzes und der ID des
Tokens, welches als Kopf bestimmt worden ist. -- Auch dies kann als
eine Form von Annotationen aufgefasst werden; sie weist eine
Ähnlichkeit mit Stand-Off-Markup%
\footnote{\Cite[\see\ z.\,B.][575--579]{TEIP5}, und insbesondere das
  Text Corpus Format (TCF) von WebLicht:
  \url{https://weblicht.sfs.uni-tuebingen.de/weblichtwiki/index.php/The_TCF_Format}.} %
auf.


\subsubsection{Wortart}

Nicht nur die Merkmalsgewichte, sondern auch die Auswertfunktionen der
Merkmale enthalten Parameter, die bestimmt werden müssen. So auch die
Auswertfunktion für das PoS-tag. Auch hier werde mit geschätzten
Parametern begonnen.  Sei $p$ die Komponente des Merkmalsvektors, in
der das PoS-tag gegeben ist, wobei Wortartenklassen des
Stuttgart-Tübingen-TagSet (STTS) \parencite{Schiller1999a} verwendet
werden, dann sei $f_{PoS}:V\mapsto [0,1]$ definiert durch:
\begin{equation}
  \label{eq:pos}
  f_{PoS}((\dots,p,\dots)) := \left\{
    \begin{array}{ll}
      1 & \mathrm{falls~} p \in \{\mathtt{NE,FM}\} \\
      0,8 & \mathrm{falls~} p \in \{\mathtt{NN}\} \\
      0,5 & \mathrm{falls~} p \in \{\mathtt{VVINF,}\\ ~ & \mathtt{VVIZU,VVPP}\} \\
      0,4 & \mathrm{falls~} p \in \{\mathtt{VVFIN}\} \\
      0,2 & \mathrm{falls~} p \in \{\mathtt{VMINF}\} \\
      0,1 & \mathrm{falls~} p \in \{\mathtt{VAINF}\} \\
      0 & \mathrm{sonst} \\
    \end{array}\right.
\end{equation}


\subsubsection{Distanz zum Marker}

Sei $d_t$ die Distanz zwischen dem untersuchten Token und dem Marker,
gemessen in dazwischenliegenden Tokens, und $l_t$ die Länge des
Satzes, in dem beide vorkommen, wieder gemessen in Tokens. Sowohl
$d_t$ als auch $l_S$ ist eine Komponente des Merkmalsvektors. Dann sei
die normierte Auswertfunktion $f_{dt}:V\mapsto [0,1]$ definiert durch:
\begin{equation}
  \label{eq:tokendistance}
  f_{dt}((\dots,d_t,l_t,\dots)):=1-\frac{d_t}{l_t} 
\end{equation}

Die Distanz in Kommata $f_{dc}$ werde analog aus $d_c$ und $l_c$
errechnet.


\subsubsection{Richtung}

Erfahrungsgemäß steht der Marker öfter vor dem Beispiel als hinter
ihm. Entsprechend enthält auch die hier sogenannte Richtung einen
Schätzwert. Seien $p_m$ und $p_t$ die Positionsnummer des Markers
bzw. des untersuchten Tokens im Satz und beides im Merkmalsvektor
gegeben. Dann sei $f_{dir}:V\mapsto [0,1]$ definiert durch:

\begin{equation}
  \label{eq:direction}
  f_{dir}((\dots,p_m,p_t,\dots)):= \left\{
    \begin{array}{ll}
      \sfrac{1}{4} & \textrm{~falls~} p_t < p_m\\
      \sfrac{3}{4} & \mathrm{~sonst} \\
    \end{array}\right.
\end{equation}

\subsubsection{Frequenz}

Die im Merkmalsvektor als $h_t$ gegebene absolute Häufigkeit des
Tokens im Dokument soll derart auf das Intervall $[0,1]$ abgebildet
werden, dass ein \latein{hapax legomenon} auf $1$ und häufigere Tokens
auf kleinere Werte abgebildet werden. Die Auswertfunktion für die
Token-Frequenz $f_{tf}:V\mapsto [0,1]$ sei definiert durch:
\begin{equation}
  \label{eq:tokenfreq}
  f_{tf}((\dots,p,\dots,h_t,H,\dots)) := \left\{
    \begin{array}{ll}
      1-c\frac{h_t-1}{H_t-1} & \textrm{falls } f_{PoS}((\dots,p,\dots)) > 0\\
      0 & \mathrm{sonst} \\
    \end{array}\right.
\end{equation}
wobei $H_t$ das Maximum der absoluten Frequenzen all der Tokens im
Dokument sei, für die $f_{PoS}>0$ gilt, die also zur offenen
Wortklasse und nicht zu den Stopp-Wörtern gehören. $H_t$ sei ebenfalls
im Merkmalsvektor gegeben. Die Funktion entspricht damit der
\englisch{augmented normalized term frequency} nach
\textcite[518]{Salton1988a}. $c$ ist dabei ein Parameter, der den
Abfall des Funktionswerts bei steigender Frequenz adjustiert.

Die Auswertfunktion für die Lemma-Frequenz $f_{fl}$ sei analog
definiert.

\subsubsection{Ergebnisse aus Phase 1}

Für die Phase 1 des zweistufigen Verfahrens ist ein Prototyp in der
Programmiersprache R implementiert. Als Gewichte werden $w_{PoS}=3$,
$w_{dt}=2$, $w_{dc}=6$, $w_{tf}=0$ und $w_{lf}=4$ verwendet.

In Immanuel Kants dritter \lit{Kritik} gibt es 51 eindeutige Marker
»z.\,B.«. Das Verfahren liefert dazu folgende Beispiel-Köpfe:%
\footnote{Datengrundlage ist hier die von Gutenberg-DE ausgelesene
  Fassung.} %
\begin{quote}
  mihi, Substanz, Bergkristall, Körper, Geister, Rose, Rasenplatzes,
  Walde, Schönheit, Grabhügeln, Tulpe, Größe, Tiere, Kunstprodukten,
  Fuß, Affekten, Gebäude, Zorn, Formen, Tulpen, Farben, Lohn,
  Dichtkunst, Kenntnis, Pferdes, Weib, Genius, Tod, Dichter, Haß,
  Leuten, Linie, Bau, tun, Parabel, Garten, Zirkels, Eigenschaft,
  Flüsse, Haus, Körper, Ungeziefer, Winde, Prädikate, Ursache, Made,
  Wassertiere, Erden, Seele, Ewigkeit, Ewigkeit
\end{quote}

Darunter sind einige Wörter, die nicht zu erwarten waren, weil es sich
um allgemeine Konzepte handelt: »Substanz«, »Körper«, »Affekten« oder
»Formen« etwa. Jedoch sind »Substanz«, »Körper« und »Affekten«
korrekt. »Formen« ist nicht korrekt und der Fehler kann auch
identifiziert werden: Er besteht darin, in $f_{dc}$ einen Doppelpunkt
nicht als Satzzeichen, nicht als Komma, gezählt zu haben. Obwohl
»Tiere« korrekt erscheint, hätte besser ein Adjektiv im selben Satz
ausgewählt werden sollen. »Schönheit« ist nicht korrekt, vielmehr
handelt es sich um das Thema der Passage -- und der halben dritten
\lit{Kritik}. Dieses Token kommt 112 Mal vor. Auch hier kann aber die
Ursache des Fehlers identifiziert werden: Ein korrekter Kandidat in
diesem Satz wäre »Wohnhause« gewesen, aber die Lemmatisierung durch
WebLicht funktioniert bei Komposita nicht richtig, sondern gibt dann
stets »unkown« aus. Weil es viele Komposita gibt, ist die Häufigkeit
dieses Lemmas entsprechend hoch und das stark gewichtete Merkmal
$f_{fl}$ niedrig. So hat »Schönheit« einen hohen Wert erlangt und ist
als Beispiel-Kopf annotiert worden.

Insgesamt handelt es sich um 5 Fehler in 51 Annotationen. Für einen
Anfang mit derart grob geschätzten Gewichten ist diese Genauigkeit
(\englisch{precision}) akzeptabel; bei der Trefferquote
(\englisch{recall}) kann man davon ausgehen, dass alle Bigramme
»z.\,B.« gefunden und zu jedem ein Kopf
bestimmt worden ist. Aber es ist klar, dass eine bessere Bestimmung
der Gewichte durch ein Regressionsverfahren wünschenswert
wäre. Allerdings wird in der Analyse der Fehler auch deutlich, wie
viel vom \englisch{pre-processing} abhängt.


In Karl Rosenkranz' \lit{Ästhetik des
  Häßlichen} \parencite{Rosenkranz1853a} gibt es 127 eindeutige
Beispielmarker. Dies ist die Liste der durch den Algorithmus
ausgewählten Köpfe: (Aus druck-technischen Gründen ist das lange S
(U+017F) durch ein normales s ersetzt worden.)

\begin{quote}
  Holzschnitten, Künstlers, lyrirische, Immermann's, Schaukel, Typus,
  Amphibien, Schaaf, Skropheln, Knochenauftreibungen, Danaë,
  Iphigenia, Danzig, Nehmen, Raums, Landschaftsmaler, Stellen,
  Betrachten, Quixote, Ornamentik, Aristophanes, Pedantismus, Stellen,
  heißt, Bertram, Index, Sonaten, Wandmalerei, Prutz, Strepsiades,
  Landschaftbilde, Platen, Architekturgemälde, Gestalten, Franzosen,
  Schwert, Maria, Cleopatra, Adam, Büffel, Ode, Frazzen, China,
  Wahrheit, Klenze's, Salzburg, Architektur, Physiognomie, Verwesung,
  Unsymmetrische, Epilepsie, Größe, Leben, Buckligter, Leben, Verse,
  Epik, Kirche, Kopperfield, Uebelmacht, Waldbrand, David, Museum,
  Benjamin, Verhältniß, Tabacksasche, Lexikon, Sport, Vaudeville,
  Reim, Romane, Hyäne, Fröschen, Menschen, Lingam, Held, Sue, Juan,
  Halm's, Kindermord, Victor, Göttin, Statue, Marmorstatue,
  Elephanten, Morolf, Bildung, sagen, Hohenbaum, Jerusalem, Koprolith,
  Heine, beweisen, Elephantiasis, Aristophanes,
  \foreignlanguage{polutonikogreek}{<arma}, %ἁϱμα, %
  Tiberius, Scene, Lüge, Kleist, Banquo, Henriade, Lear, Familie,
  Christi, Grabbe, Nase, Gregorius, Hähne, Molières, Töpfers, Sand,
  Attellanen, Hindubettler, Besitznahme, Kotzebue's, Volksbuch,
  Hauptstelle, Erdbeben, Architektur, Schenke, Unzahl, Posse, Cain.
\end{quote}

An beiden Listen wird deutlich, wie wünschenswert es wäre, nach der
Bestimmung des Kopfes eine Ausweitung auf eine Phrase zu berechnen.

\subsection{Phase 2: Entscheiden über weitere Vorkommen}

Die Ergebnisse aus Phase 1 unterstreichen, dass die Aufgabe in Phase 2
nicht trivial ist. Besonders die Tokens mittlerer Frequenz stellen
eine Herausforderung dar: Der Begriff »Körper«, der einmal korrekt als
Beispielkopf ausgewählt worden ist, kommt 33 Mal in der dritten Kritik
vor. »Größe« sogar 48 Mal. Nur ein Bruchteil der weiteren Vorkommen
stehen im Zusammenhang mit Beispielen.

Manuelle Annotationen lassen sich für die zweite Phase schnell und
ohne viel Aufwand erstellen. Auch ist das \englisch{decision tree
  learning} ein guter Kandidat für \textit{machine learning}, der zudem
vorteilhafterweise nicht so viele Trainingsdaten benötigt wie ein
Perceptron. Die größte Herausforderung liegt in dieser Phase in der
Definition hinreichend diskriminierender Merkmale. Die Frequenz, die
in Phase 1 besonders relevant gewesen ist, reicht hier höchstens dazu
aus, einen Schwellenwert zu definieren, bei dessen Überschreitung
gleich alle Vorkommen nicht annotiert werden, um im Resultat nicht so
viele falschpositive Annotationen zu haben. Hier seien lediglich
einige Ideen zu einem Merkmalsraum skizziert, deren Implementierung
noch auf sich wartet. Es sei das Dokument, in dem ein Token als
Beispielkopf ausgewählt worden ist, das Quelldokument, und der Satz
der Quellsatz; und es sei das Dokument, in dem ein weiteres Vorkommen
dieses Tokens vorliegt, das Zieldokument, und der dieses Token
enthaltende Satz der Zielsatz. Als Merkmale sind in Betracht zu ziehen:
\begin{itemize}
\item die Ähnlichkeit zwischen Quell- und Zielsatz, wobei als
  Ähnlichkeitsmaß das Vorkommen gleicher Tokens in Frage kommt;
\item die Ähnlichkeit zwischen Umgebungen, die über Quell- und
  Zielsatz hinausgehen, etwa zwei Sätze davor und dahinter, der
  gesamte Absatz oder das Kapitel;
\item durch Word-Embeddings gefundene Beziehungen zwischen Unigrammen,
  insbesondere Synonymie und Hyponymie;
\item die Ähnlichkeit von Einheiten, die kleiner sind als der Quell-
  und Zielsatz, etwa zwischen PoS-Tag-N-Grams, mit denen die
  Phrasenstruktur approximiert werden kann. Dabei wäre besonders auf
  ähnliche Modifizierer wie Adjektive zu achten;
\item eine größere absolute Häufigkeit des Tokens sowohl im Quell- als
  auch im Zieldokument muss ein Faktor sein, der die
  Wahrscheinlichkeit einer positiven Entscheidung reduziert;
\item gegebenenfalls sind Schwellenwerte für die relative Häufigkeit
  festzulegen, bei deren Überschreitung die Entscheidung bei jedem
  Vorkommen des Tokens negativ ausfällt;
\item wenn das Token mehrmals als Kopf ausgewählt worden ist, kann
  dies auch von Relevanz sein.
\end{itemize}

\section{Fazit}

Auch wenn das reformulierte Modell für Beispiel-Annotationen noch
nicht vollständig implementiert ist, zeichnet sich doch schon jetzt
mit den akzeptablen Ergebnissen der ersten Phase ab, dass ein
funktionierendes Experimentalsystem viel leichter realisierbar ist als
mit den komplexen Annotationen der propositionalen Struktur. Der Grund
ist, dass die Annotationen konzeptionell viel näher an einen
Algorithmus rücken, während dieser bei den komplexen Annotationen das
Versprechen einer Zukunft bleibt. In dem reformulierten Modell sind
die Spannungen zwischen der interpretativen Arbeit am philosophischen
Text und dem Formalismus des Vokabulars, die für die komplexen
Annotationen kennzeichnend waren, weitgehend verschwunden und es
rücken Maße wie \englisch{precision} und \englisch{recall} für die
Genauigkeit maschineller Annotationen an ihre Stelle. Entsprechend
sind auch die manuellen Annotationen nicht mehr mit großen
hermeneutischen Investitionen verbunden; Annotieren ist hier kein
\englisch{close reading} mehr, sondern tendenziell ein technischer
Vorgang. In dem beschriebenen hybriden Aufbau, in welchem die Liste
der Beispiel-Köpfe, die das regelbasierte Verfahren geliefert hat,
einer Revision unterzogen wird, um dann die Gewichte mittels
Regression besser zu bestimmen, haben die Annotationen keinerlei
Ähnlichkeit mehr mit dem Markup eines XML-Dokuments. Sie bleiben aber
dennoch deutlich eine Form des Informationsaustauschs in der
Mensch-Maschine-Kommunikation über einen Text.

\printbibliography

% Definition von Baumgarten zeigt die Komplexität

% mehrere Beispiele zu einem Marker, Beispielreihe.

% Rheinberger

\end{document}



%%% Local Variables:
%%% mode: latex
%%% TeX-master: t
%%% TeX-PDF-mode: t
%%% End:

%  LocalWords:  annotieren Humanities Schwellenwerte
